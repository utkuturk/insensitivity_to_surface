\documentclass[11pt]{letter}
\usepackage[utf8]{inputenc}
\usepackage[a4paper, margin=0.8in]{geometry}

% \name{Utku T\"{u}rk}
% Address moved to bottom as per user preference
\date{\today} 

\begin{document}

\begin{letter}{Dear Prof. Martin,}

\opening{I would like to submit my work titled \emph{``(In)sensitivity to surface-level heuristics: A case from Turkish verbal attractors''} for consideration as a research article in \emph{Cognition}.}

There is a theoretically provocative finding in the recent literature: Slioussar (2018) argues that accidental phonological resemblance can mediate memory search during dependency resolution. Specifically, she demonstrates that Russian Genitive Singulars, which are phonologically ambiguous with Nominative Plurals, induce more attraction errors than unambiguous Genitive Plurals. While it is well known that phonological overlap facilitates reading times via priming, the claim that it actively guides the retrieval mechanism for syntactic dependencies constitutes a significant departure from standard structural accounts of agreement.

However, an alternative account relies on the statistical distribution of potential controllers in a language. Researchers such as Bhatia and Dillon (2022) and Bleotu and Dillon (2024), along with Lago et al. (2015) argue that the probability of a specific case or form functioning as a subject modulates its retrieval interference. Under this view, the Russian effects may stem from the fact that the Russian genitives can be subjects in certain constructions and able to control agreement with modifiers, rather than from phonology per se.

In the enclosed manuscript, I test this proposal using Turkish, a language that allows for a critical dissociation: verbal subjects carry the plural marker \textit{-lAr}---providing both phonological overlap and structural subjecthood---yet they cannot control verbal agreement. I investigate whether attraction is triggered by these verbal attractors or only by nominal attractors (which have overlap, subjecthood, and control potential).

I report a clear dissociation: verbal attractors did not induce attraction, whereas genitive plural nouns did. This finding suggests that phonological modulation does not affect dependency resolution. Instead, I argue that the effects observed in Slioussar (2018) likely stem from the statistical distribution of being a possible controller rather than phonology itself. This analysis offers a unified explanation for why purely phonological effects are absent in Turkish and have failed to replicate in related languages like Czech, where genitives do not serve as possible agreement controllers in these contexts.

These results constrain retrieval models by demonstrating that memory access for agreement is gated by abstract controller potential, not by surface-level phonology. I believe this work makes a timely contribution to the growing literature on syncretism, surface level heuristics, and statistical information that comprehenders can utilize.

\closing{Sincerely,\\
\vspace{0.5cm}
Utku T\"{u}rk\\
University of Maryland, College Park\\
Department of Linguistics\\
Marie Mount Hall\\
College Park, 20742, USA}

\end{letter}
\end{document}
