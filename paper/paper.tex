% Options for packages loaded elsewhere
% Options for packages loaded elsewhere
\PassOptionsToPackage{unicode}{hyperref}
\PassOptionsToPackage{hyphens}{url}
\PassOptionsToPackage{dvipsnames,svgnames,x11names}{xcolor}
%
\documentclass[
  authoryear,
  preprint]{elsarticle}
\usepackage{xcolor}
\usepackage{amsmath,amssymb}
\setcounter{secnumdepth}{5}
\usepackage{iftex}
\ifPDFTeX
  \usepackage[T1]{fontenc}
  \usepackage[utf8]{inputenc}
  \usepackage{textcomp} % provide euro and other symbols
\else % if luatex or xetex
  \usepackage{unicode-math} % this also loads fontspec
  \defaultfontfeatures{Scale=MatchLowercase}
  \defaultfontfeatures[\rmfamily]{Ligatures=TeX,Scale=1}
\fi
\usepackage{lmodern}
\ifPDFTeX\else
  % xetex/luatex font selection
\fi
% Use upquote if available, for straight quotes in verbatim environments
\IfFileExists{upquote.sty}{\usepackage{upquote}}{}
\IfFileExists{microtype.sty}{% use microtype if available
  \usepackage[]{microtype}
  \UseMicrotypeSet[protrusion]{basicmath} % disable protrusion for tt fonts
}{}
\makeatletter
\@ifundefined{KOMAClassName}{% if non-KOMA class
  \IfFileExists{parskip.sty}{%
    \usepackage{parskip}
  }{% else
    \setlength{\parindent}{0pt}
    \setlength{\parskip}{6pt plus 2pt minus 1pt}}
}{% if KOMA class
  \KOMAoptions{parskip=half}}
\makeatother
% Make \paragraph and \subparagraph free-standing
\makeatletter
\ifx\paragraph\undefined\else
  \let\oldparagraph\paragraph
  \renewcommand{\paragraph}{
    \@ifstar
      \xxxParagraphStar
      \xxxParagraphNoStar
  }
  \newcommand{\xxxParagraphStar}[1]{\oldparagraph*{#1}\mbox{}}
  \newcommand{\xxxParagraphNoStar}[1]{\oldparagraph{#1}\mbox{}}
\fi
\ifx\subparagraph\undefined\else
  \let\oldsubparagraph\subparagraph
  \renewcommand{\subparagraph}{
    \@ifstar
      \xxxSubParagraphStar
      \xxxSubParagraphNoStar
  }
  \newcommand{\xxxSubParagraphStar}[1]{\oldsubparagraph*{#1}\mbox{}}
  \newcommand{\xxxSubParagraphNoStar}[1]{\oldsubparagraph{#1}\mbox{}}
\fi
\makeatother


\usepackage{longtable,booktabs,array}
\usepackage{calc} % for calculating minipage widths
% Correct order of tables after \paragraph or \subparagraph
\usepackage{etoolbox}
\makeatletter
\patchcmd\longtable{\par}{\if@noskipsec\mbox{}\fi\par}{}{}
\makeatother
% Allow footnotes in longtable head/foot
\IfFileExists{footnotehyper.sty}{\usepackage{footnotehyper}}{\usepackage{footnote}}
\makesavenoteenv{longtable}
\usepackage{graphicx}
\makeatletter
\newsavebox\pandoc@box
\newcommand*\pandocbounded[1]{% scales image to fit in text height/width
  \sbox\pandoc@box{#1}%
  \Gscale@div\@tempa{\textheight}{\dimexpr\ht\pandoc@box+\dp\pandoc@box\relax}%
  \Gscale@div\@tempb{\linewidth}{\wd\pandoc@box}%
  \ifdim\@tempb\p@<\@tempa\p@\let\@tempa\@tempb\fi% select the smaller of both
  \ifdim\@tempa\p@<\p@\scalebox{\@tempa}{\usebox\pandoc@box}%
  \else\usebox{\pandoc@box}%
  \fi%
}
% Set default figure placement to htbp
\def\fps@figure{htbp}
\makeatother





\setlength{\emergencystretch}{3em} % prevent overfull lines

\providecommand{\tightlist}{%
  \setlength{\itemsep}{0pt}\setlength{\parskip}{0pt}}



 
\usepackage[]{natbib}
\bibliographystyle{elsarticle-harv}


\makeatletter
\@ifpackageloaded{caption}{}{\usepackage{caption}}
\AtBeginDocument{%
\ifdefined\contentsname
  \renewcommand*\contentsname{Table of contents}
\else
  \newcommand\contentsname{Table of contents}
\fi
\ifdefined\listfigurename
  \renewcommand*\listfigurename{List of Figures}
\else
  \newcommand\listfigurename{List of Figures}
\fi
\ifdefined\listtablename
  \renewcommand*\listtablename{List of Tables}
\else
  \newcommand\listtablename{List of Tables}
\fi
\ifdefined\figurename
  \renewcommand*\figurename{Figure}
\else
  \newcommand\figurename{Figure}
\fi
\ifdefined\tablename
  \renewcommand*\tablename{Table}
\else
  \newcommand\tablename{Table}
\fi
}
\@ifpackageloaded{float}{}{\usepackage{float}}
\floatstyle{ruled}
\@ifundefined{c@chapter}{\newfloat{codelisting}{h}{lop}}{\newfloat{codelisting}{h}{lop}[chapter]}
\floatname{codelisting}{Listing}
\newcommand*\listoflistings{\listof{codelisting}{List of Listings}}
\makeatother
\makeatletter
\makeatother
\makeatletter
\@ifpackageloaded{caption}{}{\usepackage{caption}}
\@ifpackageloaded{subcaption}{}{\usepackage{subcaption}}
\makeatother
\journal{Cognition}
\usepackage{bookmark}
\IfFileExists{xurl.sty}{\usepackage{xurl}}{} % add URL line breaks if available
\urlstyle{same}
\hypersetup{
  pdftitle={Sensitivity to within-experiment statistics: A case from Turkish agreement attraction},
  pdfauthor={Utku Turk},
  pdfkeywords={keyword1, keyword2},
  colorlinks=true,
  linkcolor={blue},
  filecolor={Maroon},
  citecolor={Blue},
  urlcolor={Blue},
  pdfcreator={LaTeX via pandoc}}


\setlength{\parindent}{6pt}
\begin{document}

\begin{frontmatter}
\title{Sensitivity to within-experiment statistics: A case from Turkish
agreement attraction \\\large{Within-experiment statistics in agreement
attraction} }
\author[1]{Utku Turk%
\corref{cor1}%
}
 \ead{utkuturk@umd.edu} 

\affiliation[1]{organization={University of Maryland, College
Park, Linguistics},addressline={Marie Mount Hall},city={College
Park},postcode={20742},postcodesep={}}

\cortext[cor1]{Corresponding author}

        
\begin{abstract}
Surface level does not affect it, but within-experiment statistics
effect the findings.
\end{abstract}





\begin{keyword}
    keyword1 \sep 
    keyword2
\end{keyword}
\end{frontmatter}
    

\section{Introduction}\label{introduction}

Speakers often rely on additional sources of information when processing
sentences, including distributional expectations about forms and tasks,
as well as the overall composition of an experimental session (e.g., the
ratio of fillers to critical items). Recent work has demonstrated that
such task-specific factors can substantially modulate reading and
judgment behavior (e.g., Malsburg, Logacev; Hammerly et al.; Arehalli).
One line of research has used the agreement-attraction phenomenon to
probe the heuristics that influence sentence processing. Agreement
attraction refers to cases in which a verb erroneously agrees with a
nearby noun rather than the true subject, giving rise to so-called
grammaticality illusions in both production and comprehension.

\begin{enumerate}
\def\labelenumi{\arabic{enumi}.}
\tightlist
\item
  *The key to the cabinets are rusty.
\end{enumerate}

Recent experiments show that even small changes in task expectations can
alter attraction patterns. For example, Malsburg and colleagues found
that varying the practice structure and task demands (reading
vs.~judgment) affected reading times at the verb in sentences such as
the following:

\begin{enumerate}
\def\labelenumi{\arabic{enumi}.}
\setcounter{enumi}{1}
\tightlist
\item
  The singer that the actor openly admires apparently received broad
  international recognition.
\item
  The singers that the actor openly admires apparently received broad
  international recognition.
\end{enumerate}

In a self-paced reading task, when participants answered a comprehension
question after each trial, reading times at the verb did not differ
between (2) and (3). However, when participants were asked to judge
grammaticality instead, they spent more time reading the verb admires in
(3), suggesting that processing heuristics depend on the expected output
of the task.

A related set of findings comes from work by Slioussar, who showed that
surface form can sometimes override abstract features in Russian.
Exploiting the syncretism between singular genitives and nominative
plurals---a pattern absent in plural genitives---she found that
participants made more errors and showed longer reading times in
sentences like (4) than in (5). She argued that, rather than accessing
abstract case features, readers relied on surface-level cues that were
easier to retrieve.

\begin{enumerate}
\def\labelenumi{\arabic{enumi}.}
\setcounter{enumi}{3}
\tightlist
\item
  {[}Example sentence with syncretic form \ldots{]}
\item
  {[}Example sentence with non-syncretic form \ldots{]}
\end{enumerate}

Building on these observations, we utilize Turkish as a testing ground
to examine how surface-form overlap influences agreement processing and
whether exposure to different kinds of distractors modulates attraction.
Turkish provides an especially informative case because both nominal and
verbal plural markers are realized with the same morpheme, --lAr, yet
only nominal plurals bear the syntactic features required for agreement.
This allows us to ask whether participants rely on surface-form
similarity or on abstract feature representations when evaluating
agreement.

In our first experiment, we test whether plural marking on a verbal
distractor---which is morphologically identical but syntactically
irrelevant---can elicit attraction. In the second experiment, we combine
these verbal distractor conditions with standard nominal attractor
conditions to assess how their co-occurrence affects participants'
judgments. If attraction effects reflect flexible, context-sensitive
processing, the inclusion of verbal distractors should dilute or
eliminate the illusion typically observed with nominal attractors.

Together, these experiments extend previous findings on agreement
attraction and task sensitivity in two key ways. First, they show that
surface-level overlap---even when morphologically identical---does not
by itself produce agreement attraction, indicating that participants
rely on abstract morphosyntactic features rather than phonological
forms. Second, they reveal that participants are not only influenced by
the global structure of an experiment (such as the proportion of fillers
or grammatical items) but also by the presence of other condition types
within the same task. In other words, attraction effects are attenuated
when competing, non-attracting conditions are included, suggesting that
agreement processing is dynamically tuned to the statistical context of
the experiment itself.

\subsection{Agreement attraction and its
modulation}\label{agreement-attraction-and-its-modulation}

\begin{itemize}
\item
  Core phenomenon and theoretical accounts. Briefly summarize leading
  explanations:
\item
  Cue-based retrieval / similarity-based interference (Wagers, Lau \&
  Phillips 2009; Dillon et al.~2013): attraction arises because plural
  attractors partially match number cues used in retrieval.
\item
  Feature-percolation accounts (Bock \& Miller 1991; Franck et
  al.~2002): features spread upward within noun phrases, confusing the
  parser.
\item
  Context and task modulation
\item
  Show that attraction strength varies across tasks and experimental
  context:
\item
  Task type: self-paced reading vs.~acceptability judgment (Malsburg).
\item
  Response expectations and bias: Hammerly et al.~2019 --- participants'
  error patterns change when they expect more ungrammatical items.
\item
  Item composition: Laurinavichyute \& von der Malsburg 2023 ---
  manipulating filler ratios or condition mixes alters accuracy
  patterns.
\item
  Conclude that participants adapt their behavior to the statistical
  environment of the experiment; attraction is not fixed but
  context-sensitive.
\item
  Form-based influences. in some languages, surface-form similarity can
  exacerbate attraction or even drive illusions on its own.
\item
  Cite Slioussar (2018) and related Russian work: case syncretism
  between nominative plurals and genitive singulars increased attraction
  or slowed reading, suggesting reliance on form-based heuristics rather
  than abstract case features.
\item
  Cite Chromy and checz data. tell it is not very consistent.
\end{itemize}

open question --- when morphology overlaps across categories, does
attraction arise from form or feature?

\subsection{Background on Turkish}\label{background-on-turkish}

\begin{itemize}
\item
  Morphological properties
\item
  Turkish marks number on both nouns and verbs using the identical
  plural morpheme --lAr.
\item
  Only nominal plurals introduce number features that can agree with the
  verb; verbal --lAr expresses verbal agreement but is not a potential
  controller.
\item
  Because of this homophony, Turkish allows form-overlap and
  feature-mismatch to be disentangled experimentally.
\item
  Previous attraction findings in Turkish
\item
  Prior work has reported typical attraction effects with
  genitive-marked nominal attractors, showing higher acceptance of
  ungrammatical plural-verb sentences.
\item
  However, no work has directly tested whether verbal plural morphology
  can induce similar illusions, or how mixing different attractor types
  within an experiment affects the magnitude of attraction.
\end{itemize}

\subsection{Experimental logic and
predictions}\label{experimental-logic-and-predictions}

\begin{itemize}
\item
  Goal 1: test whether purely form-based overlap (verbal --lAr) elicits
  attraction.
\item
  Prediction: if attraction is driven by form, verbal plural distractors
  should yield higher ``acceptable'' rates for ungrammatical plurals.
\item
  Alternative: if attraction depends on abstract features, no effect of
  verbal --lAr should appear.
\item
  Goal 2: test whether the co-occurrence of different attractor types
  modulates attraction.
\item
  Prediction: if participants adapt to the distribution of conditions,
  adding verbal distractors (which share the plural form but lack
  agreement features) should attenuate or eliminate the
  nominal-attractor illusion.
\item
  Summary: These experiments jointly test whether agreement attraction
  in Turkish reflects shallow form matching or feature-based computation
  that is sensitive to the statistical context of the task.
\end{itemize}

\section{Experiment 1: Testing Form-Driven
Processing}\label{experiment-1-testing-form-driven-processing}

\begin{itemize}
\tightlist
\item
  Goal: determine if surface plural forms (verbal -lAr) elicit
  illusionary agreement.
\item
  Participants: 80 Turkish speakers (Boğaziçi undergraduates).
\item
  Design: 2 × 2 (Grammaticality × Attractor Number).
\item
  Materials: relative-clause verbs as attractors; same surface
  morphology as nominal plurals.
\item
  Procedure: speeded acceptability judgments, 1500 ms deadline.
\item
  Analysis: Bayesian probit GLM (brms); random intercepts/slopes by
  subject/item.
\item
  Results:

  \begin{itemize}
  \tightlist
  \item
    High filler accuracy (\textgreater{} .9).
  \item
    No difference in ungrammatical sentences between plural vs singular
    attractors.
  \item
    Posterior coefficients near 0; 95 \% CIs within ROPE.
  \end{itemize}
\item
  Discussion:

  \begin{itemize}
  \tightlist
  \item
    No evidence for form-driven guessing.
  \item
    Participants rely on abstract number features, not phonological
    similarity.
  \end{itemize}
\end{itemize}

\section{Experiment 2: Testing Within-Experiment Statistical
Sensitivity}\label{experiment-2-testing-within-experiment-statistical-sensitivity}

\begin{itemize}
\tightlist
\item
  Goal: test whether attraction changes when both attractor types occur
  in one experiment.
\item
  Participants: 95 Turkish speakers.
\item
  Design: 2 × 2 × 2 (Grammaticality × Attractor Number × Attractor Type
  {[}nominal vs verbal{]}).
\item
  Procedure \& analysis: same as Experiment 1.
\item
  Results:

  \begin{itemize}
  \tightlist
  \item
    Attraction replicated for nominal attractors (Δ ≈ 0.07).
  \item
    Verbal attractors again showed null effect.
  \item
    Global decline in yes-responses relative to earlier studies →
    participants became more conservative.
  \end{itemize}
\item
  Discussion:

  \begin{itemize}
  \tightlist
  \item
    Exposure to verbal conditions reduced attraction magnitude overall.
  \item
    Indicates participants adapt to statistical properties of the task.
  \item
    Aligns with learning-based cue-weighting accounts (Haskell et
    al.~2010).
  \end{itemize}
\end{itemize}

\section{General Discussion}\label{general-discussion}

\begin{itemize}
\tightlist
\item
  Synthesis:

  \begin{itemize}
  \tightlist
  \item
    No evidence for surface-form matching; effects are feature-based.
  \item
    Attraction magnitude changes with condition distribution → adaptive
    tuning.
  \end{itemize}
\item
  Interpretation:

  \begin{itemize}
  \tightlist
  \item
    Supports an adaptive parser sensitive to within-experiment
    statistics.
  \item
    Challenges ``shallow'' or ``good-enough'' accounts that attribute
    attraction to phonological overlap.
  \end{itemize}
\item
  Broader implication:

  \begin{itemize}
  \tightlist
  \item
    Agreement processing is flexible and probabilistic; illusions arise
    from learned cue validity.
  \end{itemize}
\item
  Limitations:

  \begin{itemize}
  \tightlist
  \item
    Syntactic depth asymmetry (verbal attractors more embedded).
  \item
    Need future designs equating structure (e.g., embedded-object
    attractors).
  \end{itemize}
\item
  Conclusion:

  \begin{itemize}
  \tightlist
  \item
    Turkish attraction effects arise from abstract feature retrieval not
    surface level shallow form-matching.\\
  \item
    The evaluation of abstract features are modulated by distributional
    learning within the experiment.
  \end{itemize}
\end{itemize}


\bibliography{bibliography.bib}



\end{document}
