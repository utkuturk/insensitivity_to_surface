% Options for packages loaded elsewhere
% Options for packages loaded elsewhere
\PassOptionsToPackage{unicode}{hyperref}
\PassOptionsToPackage{hyphens}{url}
\PassOptionsToPackage{dvipsnames,svgnames,x11names}{xcolor}
%
\documentclass[
  authoryear,
  preprint]{elsarticle}
\usepackage{xcolor}
\usepackage{amsmath,amssymb}
\setcounter{secnumdepth}{5}
\usepackage{iftex}
\ifPDFTeX
  \usepackage[T1]{fontenc}
  \usepackage[utf8]{inputenc}
  \usepackage{textcomp} % provide euro and other symbols
\else % if luatex or xetex
  \usepackage{unicode-math} % this also loads fontspec
  \defaultfontfeatures{Scale=MatchLowercase}
  \defaultfontfeatures[\rmfamily]{Ligatures=TeX,Scale=1}
\fi
\usepackage{lmodern}
\ifPDFTeX\else
  % xetex/luatex font selection
\fi
% Use upquote if available, for straight quotes in verbatim environments
\IfFileExists{upquote.sty}{\usepackage{upquote}}{}
\IfFileExists{microtype.sty}{% use microtype if available
  \usepackage[]{microtype}
  \UseMicrotypeSet[protrusion]{basicmath} % disable protrusion for tt fonts
}{}
\makeatletter
\@ifundefined{KOMAClassName}{% if non-KOMA class
  \IfFileExists{parskip.sty}{%
    \usepackage{parskip}
  }{% else
    \setlength{\parindent}{0pt}
    \setlength{\parskip}{6pt plus 2pt minus 1pt}}
}{% if KOMA class
  \KOMAoptions{parskip=half}}
\makeatother
% Make \paragraph and \subparagraph free-standing
\makeatletter
\ifx\paragraph\undefined\else
  \let\oldparagraph\paragraph
  \renewcommand{\paragraph}{
    \@ifstar
      \xxxParagraphStar
      \xxxParagraphNoStar
  }
  \newcommand{\xxxParagraphStar}[1]{\oldparagraph*{#1}\mbox{}}
  \newcommand{\xxxParagraphNoStar}[1]{\oldparagraph{#1}\mbox{}}
\fi
\ifx\subparagraph\undefined\else
  \let\oldsubparagraph\subparagraph
  \renewcommand{\subparagraph}{
    \@ifstar
      \xxxSubParagraphStar
      \xxxSubParagraphNoStar
  }
  \newcommand{\xxxSubParagraphStar}[1]{\oldsubparagraph*{#1}\mbox{}}
  \newcommand{\xxxSubParagraphNoStar}[1]{\oldsubparagraph{#1}\mbox{}}
\fi
\makeatother


\usepackage{longtable,booktabs,array}
\usepackage{calc} % for calculating minipage widths
% Correct order of tables after \paragraph or \subparagraph
\usepackage{etoolbox}
\makeatletter
\patchcmd\longtable{\par}{\if@noskipsec\mbox{}\fi\par}{}{}
\makeatother
% Allow footnotes in longtable head/foot
\IfFileExists{footnotehyper.sty}{\usepackage{footnotehyper}}{\usepackage{footnote}}
\makesavenoteenv{longtable}
\usepackage{graphicx}
\makeatletter
\newsavebox\pandoc@box
\newcommand*\pandocbounded[1]{% scales image to fit in text height/width
  \sbox\pandoc@box{#1}%
  \Gscale@div\@tempa{\textheight}{\dimexpr\ht\pandoc@box+\dp\pandoc@box\relax}%
  \Gscale@div\@tempb{\linewidth}{\wd\pandoc@box}%
  \ifdim\@tempb\p@<\@tempa\p@\let\@tempa\@tempb\fi% select the smaller of both
  \ifdim\@tempa\p@<\p@\scalebox{\@tempa}{\usebox\pandoc@box}%
  \else\usebox{\pandoc@box}%
  \fi%
}
% Set default figure placement to htbp
\def\fps@figure{htbp}
\makeatother





\setlength{\emergencystretch}{3em} % prevent overfull lines

\providecommand{\tightlist}{%
  \setlength{\itemsep}{0pt}\setlength{\parskip}{0pt}}



 
\usepackage[]{natbib}
\bibliographystyle{elsarticle-harv}


\makeatletter
\@ifpackageloaded{caption}{}{\usepackage{caption}}
\AtBeginDocument{%
\ifdefined\contentsname
  \renewcommand*\contentsname{Table of contents}
\else
  \newcommand\contentsname{Table of contents}
\fi
\ifdefined\listfigurename
  \renewcommand*\listfigurename{List of Figures}
\else
  \newcommand\listfigurename{List of Figures}
\fi
\ifdefined\listtablename
  \renewcommand*\listtablename{List of Tables}
\else
  \newcommand\listtablename{List of Tables}
\fi
\ifdefined\figurename
  \renewcommand*\figurename{Figure}
\else
  \newcommand\figurename{Figure}
\fi
\ifdefined\tablename
  \renewcommand*\tablename{Table}
\else
  \newcommand\tablename{Table}
\fi
}
\@ifpackageloaded{float}{}{\usepackage{float}}
\floatstyle{ruled}
\@ifundefined{c@chapter}{\newfloat{codelisting}{h}{lop}}{\newfloat{codelisting}{h}{lop}[chapter]}
\floatname{codelisting}{Listing}
\newcommand*\listoflistings{\listof{codelisting}{List of Listings}}
\makeatother
\makeatletter
\makeatother
\makeatletter
\@ifpackageloaded{caption}{}{\usepackage{caption}}
\@ifpackageloaded{subcaption}{}{\usepackage{subcaption}}
\makeatother
\journal{Cognition}
\usepackage{bookmark}
\IfFileExists{xurl.sty}{\usepackage{xurl}}{} % add URL line breaks if available
\urlstyle{same}
\hypersetup{
  pdftitle={Sensitivity to within-experiment statistics: A case from Turkish agreement attraction},
  pdfauthor={Utku Turk},
  pdfkeywords={keyword1, keyword2},
  colorlinks=true,
  linkcolor={blue},
  filecolor={Maroon},
  citecolor={Blue},
  urlcolor={Blue},
  pdfcreator={LaTeX via pandoc}}


\setlength{\parindent}{6pt}
\begin{document}

\begin{frontmatter}
\title{Sensitivity to within-experiment statistics: A case from Turkish
agreement attraction \\\large{Within-experiment statistics in agreement
attraction} }
\author[1]{Utku Turk%
\corref{cor1}%
}
 \ead{utkuturk@umd.edu} 

\affiliation[1]{organization={University of Maryland, College
Park, Linguistics},addressline={Marie Mount Hall},city={College
Park},postcode={20742},postcodesep={}}

\cortext[cor1]{Corresponding author}

        
\begin{abstract}
This is the abstract. Lorem ipsum dolor sit amet, consectetur adipiscing
elit. Vestibulum augue turpis, dictum non malesuada a, volutpat eget
velit. Nam placerat turpis purus, eu tristique ex tincidunt et. Mauris
sed augue eget turpis ultrices tincidunt. Sed et mi in leo porta
egestas. Aliquam non laoreet velit. Nunc quis ex vitae eros aliquet
auctor nec ac libero. Duis laoreet sapien eu mi luctus, in bibendum leo
molestie. Sed hendrerit diam diam, ac dapibus nisl volutpat vitae.
Aliquam bibendum varius libero, eu efficitur justo rutrum at. Sed at
tempus elit.
\end{abstract}





\begin{keyword}
    keyword1 \sep 
    keyword2
\end{keyword}
\end{frontmatter}
    

\section{Introduction}\label{introduction}

\begin{itemize}
\tightlist
\item
  Background: Agreement attraction refers to cases where verbs
  incorrectly agree with a nearby noun rather than the true subject,
  producing ``grammaticality illusions'' in both production and
  comprehension tasks.
\item
  Cross-linguistic picture: Attraction magnitudes vary widely across
  tasks and exposure contexts, sometimes depending on participants'
  expectations or response biases within an experiment (e.g., Hammerly
  et al., 2019; Laurinavichyute \& von der Malsburg, 2023).
\item
  Turkish as a test case: Turkish number agreement offers a unique
  window into this issue because the plural morpheme --lAr is
  homophonous across nouns and verbs. This overlap allows us to test
  whether attraction effects stem from surface-form similarity or
  abstract morphosyntactic features.
\item
  Current motivation: In many experiments, the distribution of
  conditions may itself shape how participants respond---when certain
  types of ungrammatical sentences are more frequent, people may adjust
  their acceptance patterns. Such within-experiment statistical
  sensitivity could inflate or suppress apparent attraction effects.
\item
  Goals of this study:

  \begin{enumerate}
  \def\labelenumi{\arabic{enumi}.}
  \tightlist
  \item
    Form-driven test: Determine whether the presence of a plural
    morpheme on a verbal attractor (sharing surface form with nominal
    plurals) triggers attraction.
  \item
    Statistical-sensitivity test: Examine whether attraction magnitude
    changes when different attractor types (nominal vs.~verbal) co-occur
    in the same experiment.
  \end{enumerate}
\item
  Overview: Two speeded acceptability-judgment experiments address these
  questions using Bayesian hierarchical modeling of response patterns.
\end{itemize}

\section{Experiment 1: Testing Form-Driven
Processing}\label{experiment-1-testing-form-driven-processing}

\begin{itemize}
\tightlist
\item
  Goal: determine if surface plural forms (verbal -lAr) elicit
  illusionary agreement.
\item
  Participants: 80 Turkish speakers (Boğaziçi undergraduates).
\item
  Design: 2 × 2 (Grammaticality × Attractor Number).
\item
  Materials: relative-clause verbs as attractors; same surface
  morphology as nominal plurals.
\item
  Procedure: speeded acceptability judgments, 1500 ms deadline.
\item
  Analysis: Bayesian probit GLM (brms); random intercepts/slopes by
  subject/item.
\item
  Results:

  \begin{itemize}
  \tightlist
  \item
    High filler accuracy (\textgreater{} .9).
  \item
    No difference in ungrammatical sentences between plural vs singular
    attractors.
  \item
    Posterior coefficients near 0; 95 \% CIs within ROPE.
  \end{itemize}
\item
  Discussion:

  \begin{itemize}
  \tightlist
  \item
    No evidence for form-driven guessing.
  \item
    Participants rely on abstract number features, not phonological
    similarity.
  \end{itemize}
\end{itemize}

\section{Experiment 2: Testing Within-Experiment Statistical
Sensitivity}\label{experiment-2-testing-within-experiment-statistical-sensitivity}

\begin{itemize}
\tightlist
\item
  Goal: test whether attraction changes when both attractor types occur
  in one experiment.
\item
  Participants: 95 Turkish speakers.
\item
  Design: 2 × 2 × 2 (Grammaticality × Attractor Number × Attractor Type
  {[}nominal vs verbal{]}).
\item
  Procedure \& analysis: same as Experiment 1.
\item
  Results:

  \begin{itemize}
  \tightlist
  \item
    Attraction replicated for nominal attractors (Δ ≈ 0.07).
  \item
    Verbal attractors again showed null effect.
  \item
    Global decline in yes-responses relative to earlier studies →
    participants became more conservative.
  \end{itemize}
\item
  Discussion:

  \begin{itemize}
  \tightlist
  \item
    Exposure to verbal conditions reduced attraction magnitude overall.
  \item
    Indicates participants adapt to statistical properties of the task.
  \item
    Aligns with learning-based cue-weighting accounts (Haskell et
    al.~2010).
  \end{itemize}
\end{itemize}

\section{General Discussion}\label{general-discussion}

\begin{itemize}
\tightlist
\item
  Synthesis:

  \begin{itemize}
  \tightlist
  \item
    No evidence for surface-form matching; effects are feature-based.
  \item
    Attraction magnitude changes with condition distribution → adaptive
    tuning.
  \end{itemize}
\item
  Interpretation:

  \begin{itemize}
  \tightlist
  \item
    Supports an adaptive parser sensitive to within-experiment
    statistics.
  \item
    Challenges ``shallow'' or ``good-enough'' accounts that attribute
    attraction to phonological overlap.
  \end{itemize}
\item
  Broader implication:

  \begin{itemize}
  \tightlist
  \item
    Agreement processing is flexible and probabilistic; illusions arise
    from learned cue validity.
  \end{itemize}
\item
  Limitations:

  \begin{itemize}
  \tightlist
  \item
    Syntactic depth asymmetry (verbal attractors more embedded).
  \item
    Need future designs equating structure (e.g., embedded-object
    attractors).
  \end{itemize}
\item
  Conclusion:

  \begin{itemize}
  \tightlist
  \item
    Turkish attraction effects arise from abstract feature retrieval not
    surface level shallow form-matching.\\
  \item
    The evaluation of abstract features are modulated by distributional
    learning within the experiment.
  \end{itemize}
\end{itemize}


\bibliography{bibliography.bib}



\end{document}
