% Options for packages loaded elsewhere
\PassOptionsToPackage{unicode}{hyperref}
\PassOptionsToPackage{hyphens}{url}
\PassOptionsToPackage{dvipsnames,svgnames,x11names}{xcolor}
%
\documentclass[
  authoryear,
  3p]{elsarticle}

\usepackage{amsmath,amssymb}
\usepackage{iftex}
\ifPDFTeX
  \usepackage[T1]{fontenc}
  \usepackage[utf8]{inputenc}
  \usepackage{textcomp} % provide euro and other symbols
\else % if luatex or xetex
  \usepackage{unicode-math}
  \defaultfontfeatures{Scale=MatchLowercase}
  \defaultfontfeatures[\rmfamily]{Ligatures=TeX,Scale=1}
\fi
\usepackage{lmodern}
\ifPDFTeX\else  
    % xetex/luatex font selection
\fi
% Use upquote if available, for straight quotes in verbatim environments
\IfFileExists{upquote.sty}{\usepackage{upquote}}{}
\IfFileExists{microtype.sty}{% use microtype if available
  \usepackage[]{microtype}
  \UseMicrotypeSet[protrusion]{basicmath} % disable protrusion for tt fonts
}{}
\makeatletter
\@ifundefined{KOMAClassName}{% if non-KOMA class
  \IfFileExists{parskip.sty}{%
    \usepackage{parskip}
  }{% else
    \setlength{\parindent}{0pt}
    \setlength{\parskip}{6pt plus 2pt minus 1pt}}
}{% if KOMA class
  \KOMAoptions{parskip=half}}
\makeatother
\usepackage{xcolor}
\setlength{\emergencystretch}{3em} % prevent overfull lines
\setcounter{secnumdepth}{5}
% Make \paragraph and \subparagraph free-standing
\makeatletter
\ifx\paragraph\undefined\else
  \let\oldparagraph\paragraph
  \renewcommand{\paragraph}{
    \@ifstar
      \xxxParagraphStar
      \xxxParagraphNoStar
  }
  \newcommand{\xxxParagraphStar}[1]{\oldparagraph*{#1}\mbox{}}
  \newcommand{\xxxParagraphNoStar}[1]{\oldparagraph{#1}\mbox{}}
\fi
\ifx\subparagraph\undefined\else
  \let\oldsubparagraph\subparagraph
  \renewcommand{\subparagraph}{
    \@ifstar
      \xxxSubParagraphStar
      \xxxSubParagraphNoStar
  }
  \newcommand{\xxxSubParagraphStar}[1]{\oldsubparagraph*{#1}\mbox{}}
  \newcommand{\xxxSubParagraphNoStar}[1]{\oldsubparagraph{#1}\mbox{}}
\fi
\makeatother


\providecommand{\tightlist}{%
  \setlength{\itemsep}{0pt}\setlength{\parskip}{0pt}}\usepackage{longtable,booktabs,array}
\usepackage{calc} % for calculating minipage widths
% Correct order of tables after \paragraph or \subparagraph
\usepackage{etoolbox}
\makeatletter
\patchcmd\longtable{\par}{\if@noskipsec\mbox{}\fi\par}{}{}
\makeatother
% Allow footnotes in longtable head/foot
\IfFileExists{footnotehyper.sty}{\usepackage{footnotehyper}}{\usepackage{footnote}}
\makesavenoteenv{longtable}
\usepackage{graphicx}
\makeatletter
\newsavebox\pandoc@box
\newcommand*\pandocbounded[1]{% scales image to fit in text height/width
  \sbox\pandoc@box{#1}%
  \Gscale@div\@tempa{\textheight}{\dimexpr\ht\pandoc@box+\dp\pandoc@box\relax}%
  \Gscale@div\@tempb{\linewidth}{\wd\pandoc@box}%
  \ifdim\@tempb\p@<\@tempa\p@\let\@tempa\@tempb\fi% select the smaller of both
  \ifdim\@tempa\p@<\p@\scalebox{\@tempa}{\usebox\pandoc@box}%
  \else\usebox{\pandoc@box}%
  \fi%
}
% Set default figure placement to htbp
\def\fps@figure{htbp}
\makeatother

\usepackage{gb4e}
\noautomath
% \usepackage[inline]{glossaries}
\usepackage{leipzig}
% \makeglossaries
\usepackage{typgloss}
\usepackage{setspace}
\usepackage{lineno}
\linenumbers
\usepackage{booktabs}
\usepackage{longtable}
\usepackage{array}
\usepackage{multirow}
\usepackage{wrapfig}
\usepackage{float}
\usepackage{colortbl}
\usepackage{pdflscape}
\usepackage{tabu}
\usepackage{threeparttable}
\usepackage{threeparttablex}
\usepackage[normalem]{ulem}
\usepackage{makecell}
\usepackage{xcolor}
\makeatletter
\@ifpackageloaded{caption}{}{\usepackage{caption}}
\AtBeginDocument{%
\ifdefined\contentsname
  \renewcommand*\contentsname{Table of contents}
\else
  \newcommand\contentsname{Table of contents}
\fi
\ifdefined\listfigurename
  \renewcommand*\listfigurename{List of Figures}
\else
  \newcommand\listfigurename{List of Figures}
\fi
\ifdefined\listtablename
  \renewcommand*\listtablename{List of Tables}
\else
  \newcommand\listtablename{List of Tables}
\fi
\ifdefined\figurename
  \renewcommand*\figurename{Figure}
\else
  \newcommand\figurename{Figure}
\fi
\ifdefined\tablename
  \renewcommand*\tablename{Table}
\else
  \newcommand\tablename{Table}
\fi
}
\@ifpackageloaded{float}{}{\usepackage{float}}
\floatstyle{ruled}
\@ifundefined{c@chapter}{\newfloat{codelisting}{h}{lop}}{\newfloat{codelisting}{h}{lop}[chapter]}
\floatname{codelisting}{Listing}
\newcommand*\listoflistings{\listof{codelisting}{List of Listings}}
\makeatother
\makeatletter
\makeatother
\makeatletter
\@ifpackageloaded{caption}{}{\usepackage{caption}}
\@ifpackageloaded{subcaption}{}{\usepackage{subcaption}}
\makeatother
\journal{Cognition}

\usepackage[]{natbib}
\bibliographystyle{elsarticle-harv}
\usepackage{bookmark}

\IfFileExists{xurl.sty}{\usepackage{xurl}}{} % add URL line breaks if available
\urlstyle{same} % disable monospaced font for URLs
\hypersetup{
  pdftitle={(In)sensitivity to surface-level heuristics: A case from Turkish verbal attractors},
  pdfauthor={Utku Turk},
  pdfkeywords={form-sensitivity, memory, agreement
attraction, linguistic illusions, sentence processing},
  colorlinks=true,
  linkcolor={blue},
  filecolor={Maroon},
  citecolor={Blue},
  urlcolor={Blue},
  pdfcreator={LaTeX via pandoc}}


\setlength{\parindent}{6pt}
\begin{document}

\begin{frontmatter}
\title{(In)sensitivity to surface-level heuristics: A case from Turkish
verbal attractors}
\author[1]{Utku Turk%
\corref{cor1}%
}
 \ead{utkuturk@umd.edu} 

\affiliation[1]{organization={University of Maryland, College
Park, Linguistics},addressline={Marie Mount Hall},city={College
Park},postcode={20742},postcodesep={}}

\cortext[cor1]{Corresponding author}

        
\begin{abstract}
Linguistic illusion literature has stimulated ongoing debate on what
type of information can be used to access memory representations. Prior
work tests whether structural, semantic, or discourse cues guide
subject-verb dependencies; it remains unclear whether native speakers
rely on phonological information as a retrieval cues for memory access
during dependency resolution, such as person agreement. Traditionally,
accidental phonological resemblance to having a plural ending as in /s/
sound in \emph{course} was found to not induce erroneous plural
agreement, meanwhile, phonological resemblance that correlates with
controllerhood amplifies attraction given an already present plural
morpheme. In apparent contradiction to this generalization, Slioussar
(2018) proposed that memory search for a subject in Russian sentences
can be mediated through an accidental phonological resemblance. Given
the theoretical importance of this proposal and the lack of comparable
effects in other languages, we test whether phonological overlap can
elicit erroneous agreement in Turkish, where the plural morpheme -lAr
surfaces on both nouns and verbs. Turkish provides a critical test: both
verbal elements and nominal elements can surface as subjects, but only
nominal plural -lAr controls verbal agreement. Two speeded acceptability
studies show no attraction from plural-marked verbs (Exp. 1 N = 80; Exp.
2 N = 95) but robust attraction from genitive plural nouns. We report a
first-of-its-kind dissociation under minimal manipulation: verbal
attractors that can be subjects yet cannot control agreement do not
induce attraction, whereas genitive plural nouns that can be subjects
and control in other environments do. To our knowledge this pattern has
not been shown in any other language, and it constrains cue-based
retrieval by tying attraction to abstract controller features rather
than surface phonology.
\end{abstract}





\begin{keyword}
    form-sensitivity \sep memory \sep agreement
attraction \sep linguistic illusions \sep 
    sentence processing
\end{keyword}
\end{frontmatter}
    

\section{Introduction}\label{introduction}

Human sentence processing draws on abstract grammatical features and on
heuristics that exploit surface regularities, such as plausibility
\citep{SpeerClifton1998}, frequency \citep{LauEtAl2007}, and
task-specific factors
\citep{LauraMalsbug24, ArehalliWittenberg2021, HammerlyEtAl2019, LogacevVasishth2016}.
We focus on one such heuristic: over-reliance on surface form, evidenced
when phonological similarity between sentence constituents is observed
to modulate performance
\citep{AchesonMacDonald2011, KushEtAl2015, CopelandRadvansky2001, RastleDavis2008}.
Prior work shows reliable slowdowns and comprehension accuracy costs due
to surface-form overlap, but it is unresolved whether this heuristic
penetrates dependency resolution itself--including subject-verb
agreement, pronoun resolution, or the licensing of negative polarity
items--beyond general effects on reading ease and memory. The few
studies that bear directly on subject-verb agreement exhibit
contradictory findings \citep{BockEberhard1993, Slioussar2018}.

A central question for understanding human cognition is what information
is encoded and later available to memory during comprehension, and how
faithful these encodings are to the input. `Good-Enough' and noisy
channel accounts argue that detailed analyses are not always maintained
when heuristics suffice, creating the opportunity for surface
regularities to affect judgments \citep{FerreiraEtAl2002}. More
specifically, general cue-based retrieval approaches hold that
constituents are stored with detailed abstract features and later
accessed by matching retrieval cues, and that erroneous parses can occur
when features conflict or interfere. However, it remains open whether
phonological codes are used as such cues during syntactic dependency
building \citep{LV05}, or even whether they persist long enough to do
so. Determining whether surface-form overlap modulates dependency
resolution provides a window into what human cognition counts as
diagnostic information for retrieving dependency controllers and how
faithful the stored representations are.

Agreement is an ideal case study because its computations are known to
be sensitive to feature overlap. Classic findings demonstrate systematic
errors in establishing number agreement between a verb and its agreement
controller when an NP with a different number (the attractor)
interferes, observed when speakers produce sentences like (\ref{og}) or
misclassify them as acceptable
\citep{BockMiller:1991, PearlmutterGarnseyBock:1999}.

\begin{exe}
\ex[*]{\label{og} The player on the courts are tired from a long-game.}
\end{exe}

Despite much research on what factors modulate agreement errors, the
role of phonology remains unclear. Pseudoplural attractors whose
phonological offset matches the plural suffix (e.g.~\emph{course}) do
not increase agreement errors in production \citep{BockEberhard1993}.
Phonological overlap effects have been observed in other cases, but many
of them involve additional shared morphological features
\citep{HartsuikerEtAl2003, LagoEtAl2019, BleotuDillon2024}, although not
all \citep{Slioussar2018}. This raises the possibility that surface form
affects the formation of agreement dependencies not directly through the
use of number form as a retrieval cue, but indirectly, when the surface
form is one that is more likely to be realized on agreement controllers.

Here we test this hypothesis by utilizing the surface-form overlap
between the verbal and nominal morphological reflexes of agreement that
happens to occur in Turkish. Turkish uses the same surface suffix,
\emph{-lAr}, for plural marking on nouns and for plural agreement on
finite verbs. Crucially, strings bearing verbal \emph{-lAr} can occur in
subject position, yet they never control finite clause agreement; only
nominal plurals do. These properties allow a direct test of whether form
overlap alone causes agreement errors, or whether form overlap effects
must be mediated by an element that can in principle serve as an
agreement controller (true of nouns but not verbs). Across two
high-powered speeded acceptability experiments in Turkish we find that
plural marking on an embedded verbal attractor does not increase
acceptance of plural agreement on the matrix verb; such effects are only
observed when the plural marker appears on a non-subject noun attractor.
These results indicate that surface-form overlap alone does not function
as a retrieval cue for agreement in Turkish. Dependency resolution
appears to rely on abstract features and structural relations, with
phonology influencing processing primarily outside of retrieval.

\subsection{Background}\label{background}

Agreement has been a central domain of investigation for language
processing research on memory. Across the world's languages,
morphological marking of agreement between a sentential verb and one or
more of its arguments---termed the agreement controller---is extremely
common; one survey reports that native speakers from 296 of out 378
languages surveyed exhibit systematic agreement between the verb and
another constituent(s) \citep{WALS}. However, this agreement process is
not always reliable. In their seminal work, \citet{BockMiller:1991}
showed that participants produce reliably more erroneous
non-controller-matching plural verb forms in English when an embedded
`attractor' noun was plural---for example, producing a plural-marked
continuation such as are instead of is occurs more often after
(\ref{true-pl}) than (\ref{true-sg}). The effect of the number
mismatching attractor, agreement attraction, has also been found to be
robust in comprehension
\citep{NicolEtAl1997, PearlmutterGarnseyBock:1999} of such sentences in
various languages, including Arabic \citep{TuckerEtAl:2015}, Armenian
\citep{AvetisyanEtAl:2020}, Hindi \citep{BhatiaDillon2022}, Spanish
\citep{LagoEtAl2015}, Russian \citep{Slioussar2018}, and Turkish
\citep{LagoEtAl2019, TurkLogacev2024, Ulusoy2023}.

\begin{exe}
\ex \label{initial}
\begin{xlist}
    \ex \label{true-sg} {Singular Attractor} \\ The {player} on the {court} \ldots{}
    \ex \label{true-pl} {Plural Attractor} \\ The {player} on the {courts} \ldots{}
\end{xlist}
\end{exe}

Many studies have investigated the various syntactic and semantic
factors which make agreement errors more or less likely. These factors
include hierarchical distance
\citep{HatsuikerEtAl2001, NicolEtAl1997, Kaan2002}, linear distance
\citep{Pearlmutter2000, BockCutting1992}, semantic interactions of nouns
involved \citep{Eberhard1999, ViglioccoEtAl95, HumphreysBock2005}, and
syntactic category of the phrase containing the attractor
\citep{BockMiller:1991, BockCutting1992}. One widely accepted set of
accounts that attempted to capture these error profiles is called
retrieval based theories \citep{LV05, WagersEtAl:2009}. In these
accounts, participants have a faithful representation of the
constituents they process, and that errors arise because they are misled
by the memory mechanisms they use to identify the agreement controller.
Under this approach, phrases are encoded in a content-addressable memory
as bundles of features called \emph{chunks} which include information
like, number, gender, and syntactic information
\citep{SmithVasishth2020}. Participants predict the number of the verb
based on the noun phrases they process while reading the previous noun
phrases. In grammatical sentences with singular verb agreement, the
number prediction and the verb number match, which causes no processing
difficulty. In contrast, when participants fail to find the predicted
number morphology on the verb, a memory-retrieval process is initiated.
This process activates the search for a chunk matching relevant cues for
agreement controller.

What is the characteristics of cues which are found useful to be
encoded? One line of work manipulated overt case marking on attractors
to test whether morphophonological case is used for dependency
resolution. For example, \citet{HartsuikerEtAl2003} used the syncretic
homophony between nominative/accusative and singular/plural forms of
feminine determiners in German, comparing these ambiguous forms to
distinctly marked dative forms. Participants produced more agreement
errors when the preambles contained two noun phrases whose determiners
were ambiguously marked (\emph{die}), compared to cases where the
attractor case could be distinguished by form alone (\emph{den}).
Furthermore, this additive effect was limited to feminine nouns, the
only gender showing nominative--accusative syncretism in plural forms,
while nouns of other grammatical genders showed the base effect of
plural.

However, results from other languages with overt case marking are more
mixed. \citet{FrankEtAl2010}, working in French, compared unambiguously
accusative-marked attractors to NPs with no overt case marking and found
that unambiguous case increased attraction, contrary to the simple
prediction that reducing ambiguity should reduce interference.
\citet{AvetisyanEtAl:2020} similarly reported that unambiguous case
marking in Armenian did not reliably modulate either reading times or
attraction errors. These findings suggest that the mere presence of
distinct case morphology is not sufficient to predict interference, and
that language specific distributions or heuristic use of case may also
be involved.

A second line of work tests phonological overlap that does not itself
change the syntactic analysis. \citet{BockEberhard1993} tested whether
attractors that only sound plural, pseudoplural singular attractors such
as \emph{course}, increase agreement errors compared to true plural
nouns, such as \emph{courts} in (\ref{true-pl}). They reasoned that if
participants rely on phonological cues rather than abstract features,
words ending with plural-like sounds (/s/ or /z/) should behave like
true plurals. In their preamble completion study, they found that
pseudoplural attractors did not induce agreement errors, which argues
against a purely phonology-driven account of attraction in English.

In contrast, \citet{Slioussar2018} reported a robust contribution of
surface-form overlap to agreement in Russian. In Russian, a subset of
genitive singular nouns is homophonous with nominative plural forms,
while genitive plural forms are not ambiguous in this way. In a series
of production and comprehension experiments, \citet{Slioussar2018}
showed that sentences with a singular genitive attractor whose form
overlaps with nominative plural yielded more plural completions, faster
reading times at the plural verb and higher rates of acceptability
compared to the sentences with unambiguous genitive plural attractors.
\citet{Slioussar2018} took these results to be an evidence for a
retrieval process in which the search for a controller is mediated
through phonological form and relevant features like +NOM and +PL can be
activated. However, mixed previous findings in case-syncretism
literature and English pseudoplural casts a shadow on this explanation.

An alternative account that does not depend on activation of relevant
features by phonology would depend on encoding of distributional facts
as statistical heuristics. In such an account, instead of relying on
activation of features through a phonological route, participants would
probabilistically associate certain strings, such as genitive marked NP
or overt D head, with being an agreement controller. Indeed, similar
explanations for syncretism or subject-likeness phenomenon has been
reported. For example, \citet{LagoEtAl2019} argued that participants can
retrieve a noun as the controller if the noun is marked with a case
marking that may sometimes control agreement in a language even if that
is not the case for the specific sentence. They used Turkish genitive
case, which can control the agreement in embedded sentences but not in
matrix sentences. They took the presence of attraction effects in
Turkish as an indication that Turkish speakers utilize overt
genitive-case's association with subjecthood. In a sense, phonological,
not functional, syncretism between the marking on the nominal modifier
and the embedded subject resulted in attraction. A similar account from
Dillon and colleagues was pushed for sensitivity for looking like a
controller in languages like Romanian and Hindi
\citep{BhatiaDillon2022, BleotuDillon2024}. For instance,
\citet{BleotuDillon2024} manipulated whether the attractor surfaces with
a determiner or in its bare form. Importantly, they note that only nouns
with determiners can control agreement in Romanian. They found that
Romanian attractors only induced attraction effects when both attractor
and the head surfaced with a determiner. They took these results to
suggest that participants associated presence of a determiner or related
feature with the agreement controller, and attraction only surfaces when
subject heads and the attractor look alike. Similarly,
\citet{SchlueterEtAl2018} argue that and can cause agreement attraction
effects in English even when it does not create a plurality because it
is associated with the plural feature statistically. Such explanations
are based on the assumption that the match between a cue and a chunk
does not have to be categorical, but it can be influenced by surface
level statistical association \citep{EngelmannEtAl2019}.

A similar account can also be proposed for Russian findings. Genitive
marked nouns can be subjects in negative inversion constructions in
Russians. However, when they are subjects, they cannot control the
agreement. In other cases, they can be the controller of number or
gender marking on adjectival relative clauses. Given this possibility of
an alternative account, the contention of initial findings of
\citet{BockEberhard1993}, and the theoretical importance of the
empirical generalization, we test a stronger version of the phonological
modulation hypothesis: whether overlap in overt plural morphology that
matches the agreement suffix in both form and plural semantics, while
being syntactically unable to serve as an agreement controller, can by
itself give rise to attraction in two high-powered speeded acceptability
judgment experiments. To this end we use Turkish, a language where
verbal and nominal plural marking share the same surface form, the
suffix \emph{--lAr}. We use reduced relative clause (RRC) structures, in
which the verb with the plural marking alone can appear as the attractor
(\ref{rrc-intro}). Importantly, Turkish \emph{--lAr} syncretism here is
not feature-ambiguous (as in cases of syncretism); it is a form-only
overlap that does not share possible argument status with a possible
controller. Even when the RRC can surface without its head as the
subject, they cannot control the agreement (\ref{rrc-subject}).

\begin{exe}
\ex \label{rrc-intro}
\gll Gör-dük-ler-i çocuk koş-tu-(*lar).\\
go-NMLZ-PL-POSS kid[NOM] run-PST-(*PL)\\
\glt `The kid that (they) saw ran.'
\ex \label{rrc-subject}
\gll Gör-dük-ler-i koş-tu-(*lar).\\
go-NMLZ-PL-POSS run-PST-(*PL)\\
\glt `(The kid) that (they) saw ran.'
\end{exe}

In Experiment 1, we tested the form hypothesis by comparing sentences
with verbal attractors to sentences with canonical nominal attractors in
Turkish. Experiment 2 then tested the form hypothesis more directly by
only using verbal attractors. We expected that if surface-overlap can
modulate relevant memory representations for dependency resolutions, we
would see similar attraction results with nominal and verbal attractors.
However, if participants are tracking an higher order cue that is
relevant for being a possible controller, then the verbal attractors,
due to their inability to control agreement, would not introduce
agreement attraction effects even though their high morpho-phonological
similarity.

Across both experiments, we found no evidence that verbal \emph{--lAr}
induces attraction, even when canonical nominal attractors are present
in the same session. This pattern aligns with prior findings in general
attraction literature and Turkish agreement attraction, namely
surface-form overlap alone does not derive agreement illusions. Rather,
attraction appears to depend on abstract feature overlap between
potential controllers and agreement probes, and possibly statistical
associations between the strings and their controllers. In this light,
findings of \citet{Slioussar2018} are best analyzed as a possible
increased association between genitive marking and possible subjecthood
and being an agreement controller. By doing so, we hope to clarify how
cue-mechanisms are employed and the role of phonological overlap in
sentence processing.

\section{Experiment 1: Testing Surface-Form
Overlap}\label{experiment-1-testing-surface-form-overlap}

\subsection{Participants}\label{participants}

We recruited 95 undergraduate students to participate in the experiment
in exchange for course credit. All participants were native Turkish
speakers, with an average age of 21 (range: 18 -- 30). The experiment
was carried out following the principles of the Declaration of Helsinki
and the regulations concerning research ethics at Bogazici University.
All participants provided informed consent before their participation
and their identities were completely anonymised.

\subsection{Materials}\label{materials}

We used 40 sets of sentences like (\ref{exp}), in which we manipulated
(i) the number of the attractor, (ii) the type of the attractor, and
(iii) the number agreement on the verb. Both plural markings were marked
with the suffix -ler/-lar, while the singular number and singular
agreement were marked by its absence.

\begin{exe}
\ex \label{exp}
\begin{xlist}
\ex[]{\label{ss}
\gll Tut-tuğ-u aşçı mutfak-ta sürekli zıpla-dı.\\
hire-NMLZ-POSS cook[NOM] kitchen-LOC non.stop jump-PST\\
\glt `The cook they hired$_{sg}$ jumped$_{sg}$ in the kitchen non-stop.'}
\ex[*]{\label{sp}
\gll Tut-tuğ-u aşçı mutfak-ta sürekli zıpla-dı-lar.\\
hire-NMLZ-POSS cook[NOM] kitchen-LOC non.stop jump-PST-PL\\
\glt `The cook they hired$_{sg}$ jumped$_{pl}$ in the kitchen non-stop.'}
\ex[]{\label{ps}
\gll Tut-tuk-lar-ı aşçı mutfak-ta sürekli zıpla-dı.\\
hire-NMLZ-PL-POSS cook[NOM] kitchen-LOC non.stop jump-PST\\
\glt `The cook they hired$_{pl}$ jumped$_{sg}$ in the kitchen non-stop.'}
\ex[*]{\label{pp}
\gll Tut-tuk-lar-ı aşçı mutfak-ta sürekli zıpla-dı-lar.\\
hire-NMLZ-PL-POSS cook[NOM] kitchen-LOC non.stop jump-PST-PL\\
\glt `The cook they hired$_{pl}$ jumped$_{pl}$ in the kitchen non-stop.'}
\ex[]{\label{nss}
\gll Milyoner-in aşçı-sı mutfak-ta sürekli zıpla-dı.\\
millionaire-GEN cook[NOM]-POSS kitchen-LOC non.stop jump-PST\\
\glt `The millionaire's cook jumped$_{sg}$ in the kitchen non-stop.'}
\ex[*]{\label{nsp}
\gll Milyoner-in aşçı-sı mutfak-ta sürekli zıpla-dı-lar.\\
millionaire-GEN cook[NOM]-POSS kitchen-LOC non.stop jump-PST-PL\\
\glt `The millionaire's cook jumped$_{pl}$ in the kitchen non-stop.'}
\ex[]{\label{nps}
\gll Milyoner-ler-in aşçı-sı mutfak-ta sürekli zıpla-dı.\\
millionaire-PL-GEN cook[NOM]-POSS kitchen-LOC non.stop jump-PST\\
\glt `The millionaires' cook jumped$_{sg}$ in the kitchen non-stop.'}
\ex[*]{\label{npp}
\gll Milyoner-ler-in aşçı-sı mutfak-ta sürekli zıpla-dı-lar.\\
millionaire-PL-GEN cook[NOM]-POSS kitchen-LOC non.stop jump-PST-PL\\
\glt `The millionaires' cook jumped$_{pl}$ in the kitchen non-stop.'}
\end{xlist}
\end{exe}

All sentences were adapted by previous studies in Turkish agreement
attraction \citep{LagoEtAl2019, TurkLogacev2024}. Sentences with verbal
attractor (\ref{ss}-\ref{pp}) started with a complex subject NP like
`tuttukları aşçı' `the cook they hired,' in which the nominalized
relative clause functioned as the attractor, and the head noun were
bare. Because the plural marking on nominals is not optional and the
head noun was singular, absent of -lar, in all conditions, sentences
with plural verb agreement were ungrammatical. In the other 4 conditions
(\ref{nss}-\ref{npp}), we simply used the items from
\citet{TurkLogacev2024}, where the attractors were nominal such as
`milyonerlerin aşçısı' `the millionaires' cook'. To inhibit participants
from forming a task-related strategy in which they deemed the sentence
ungrammatical upon seeing a plural verb, half of our fillers included
plural grammatical verbs, while the other half included singular
ungrammatical verbs.

\subsection{Procedures}\label{procedures}

The experiment was run online, using the web-based platform Ibex Farm
\citep{Drummond2013}. Each experimental session took approximately 25
minutes to complete. Participants provided demographic information and
gave informed consent to participate in the experiment. They then
proceeded to read the instructions and were given nine practice trials
before the experiment began.

Each trial began with a blank screen for 600 ms, followed by a
word-by-word RSVP presentation of the sentence in the center of the
screen, followed by a prompt to indicate their acceptability judgment.
Sentences were presented word-by-word in the center of the screen in 30
pt font size, at a rate of 400 ms per word. Participants saw a blank
screen for 100 ms between each word, and to see the next item, they
needed to press the space key. Participants were asked to press the key
P to indicate that a sentence is acceptable and Q to indicate that the
sentence is unacceptable. They were instructed to provide judgments as
quickly as possible. During the practice, but not during the experiment,
a warning message in red font appeared if they did not respond within
5,000 ms.

Participants saw 40 experimental and 40 filler sentences. Experimental
sentences were distributed among four different lists according to a
Latin-square design. Every participant saw one version of the experiment
with a specific list and one item per condition.

\subsection{Analysis and Results}\label{analysis-and-results}

Participants showed high accuracy in both grammatical (M = 0.95, CI =
{[}0.94,0.96{]}) and ungrammatical filler sentences (M = 0.06, CI =
{[}0.05,0.07{]}), indicating that they understood the task and performed
it reliably.

Figure~\ref{fig-exp2-condition-means} presents the overall means and
credible intervals for `yes' responses across experimental conditions,
as well as the previous data from \citet{TurkLogacev2024}, which is
quite similar to the magnitude of \citet{LagoEtAl2019}. As shown, in our
study, participant gave more `yes' responses to ungrammatical sentences
with plural genitive-marked nominal attractors (M = 0.88, CI =
{[}0.85,0.91{]}) compared to their singular counterparts (M = 0.88, CI =
{[}0.85,0.91{]}).

However, similar increase in acceptability was not found with relative
clause attractors (M = 0.95 and 0.95, CI = {[}0.93, 0.97{]} and {[}0.93,
0.97{]} for singular and plural attractors, respectively). Participants
rated grammatical sentences similarly independent of the attractor
number or attractor type.

\begin{figure}

\centering{

\pandocbounded{\includegraphics[keepaspectratio]{paper_files/figure-pdf/fig-exp2-condition-means-1.pdf}}

}

\caption{\label{fig-exp2-condition-means}Mean proportion of `acceptable'
responses by grammaticality, attractor number and attractor type. Error
bars show 95\% Clopper--Pearson confidence intervals.}

\end{figure}%

Our models also showed similar results, assuming a Bernoulli logit link.
Our main research question was whether verbal attractors induced
attraction effects. We also wanted to verify the cannonical attraction
effects in Turkish with nominal attractors. To that end, we included
genitive marked nominals from data from our experiment and
\citet{TurkLogacev2024}. The model was fitted to the binary
\emph{yes/no} responses and assumed uninformative priors. Grammaticality
and Attractor Number was sum coded (grammatical = 0.5, ungrammatical =
−0.5; plural = 0.5, singular = −0.5). Attractor Type (Nominal-Current,
Nominal-TL24, Verbal) was represented by two orthogonal Helmert
contrasts: an initial contrast comparing verbal attractors to the
average of the two nominal conditions (Nominal-Current = −1/6,
Nominal-TL24 = −1/6, Verbal = 1/3) and another contrast comparing the
two nominal conditions (Nominal-Current = 1/3, Nominal-TL24 = −1/3,
Verbal = 0). All fixed effects and their interaction were included,
along with random intercepts and slopes for both subjects and items.

We present posterior summaries of estimated regression effects from our
model in Figure~\ref{fig-exp2-fixed-effects}. Our model showed a robust
attraction in both nominal attractor cases, with strongly negative
effects for our nominal items (M = -1.45, CI = {[}-2.12, -0.81{]},
P(\textless0) = \textgreater0.99) and items from \citet{TurkLogacev2024}
(M = -1.17, CI = {[}-1.64, -0.69{]}, P(\textless0) = \textgreater0.99).
More importantly, our model found no evidence for an attraction in
verbal attractor conditions (M = 0.07, CI = {[}-0.74, 0.87{]},
P(\textless0) = 0.44), verifying our observations in the descriptive
statistics. We did not find an evidence for a difference in magnitude of
attraction between the two nominal-type attractors was not found (M =
-0.29, CI = {[}-1.09, 0.5{]}, P(\textless0) = 0.73), suggesting the
presence of an additional conditions did not affect attraction
magnitudes. Finally, we found strong evidence for a decreased overall
acceptability for nominal items in our experiment (M = -1.09, CI =
{[}-1.75, -0.42{]}, P(\textless0) = \textgreater0.99), suggesting the
within-experimental distribution did affect overall acceptability, but
not attraction.

\begin{figure}

\centering{

\pandocbounded{\includegraphics[keepaspectratio]{paper_files/figure-pdf/fig-exp2-fixed-effects-1.pdf}}

}

\caption{\label{fig-exp2-fixed-effects}Posterior summaries of
attraction-related effects. Points indicate posterior means, and
horizontal bars show 95\% credible intervals on the log-odds (β) scale.
Attraction was estimated as the interaction between grammaticality and
attractor number within each attractor type. Negative values indicate
stronger attraction (a reduced ungrammaticality penalty in
plural-attractor conditions). Dashed line denotes zero (no effect).}

\end{figure}%

\subsection{Discussion}\label{discussion}

In Experiment 1, we tested whether phonological overlap between nominal
and verbal plural morphemes in Turkish induces agreement attraction. The
results provided no evidence for attraction driven by surface-form
similarity. Ungrammatical sentences with plural-marked verbs were not
judged more acceptable when the relative clause verb contained a plural
morpheme. Instead, participants reliably rejected such sentences
regardless of attractor number while showing a canonical attraction
effects with nominal attractors. This indicates that the verbal plural
marker \emph{-lAr} does not create the same type of interference
observed with nominal plural attractors.

Our results and between experiment comparison showed that
within-experiment statistics, i.e.~exposure to verbal attraction
conditions attraction items, did not substantially reduced the magnitude
of the attraction effects. However, the overall acceptability in our
nominal attractor sentences were reduced compared to the trials from
\citet{TurkLogacev2024}. This is inline with previous findings that
shows participants' judgments within the experiment are modulated by the
distribution of trials. Interestingly, previous studies achieved this
with instructions or filler elements
\citep{HammerlyEtAl2019, ArehalliWittenberg2021}. We show that the
experimental conditions and the presence of an effect within a subset of
conditions also plays a role in modulating overall acceptability.

One remaining concern is that our mixed design, which combined canonical
nominal attractor items with purely phonological verbal attractor items,
might itself have shaped the pattern of responses. The presence of
robust nominal attraction trials could have led participants to adjust
their expectations about agreement violations or to adopt task
strategies that obscure any weaker effect of verbal plural markers
\citep{HammerlyEtAl2022, Turk2022}. To assess whether the absence of
verbal attraction in Experiment 1 reflects a genuine lack of
interference from verbal -lAr rather than an artifact of the item
distribution, Experiment 2 replicated the verbal attractor conditions
while removing all nominal attractor items. This design allows us to
test whether the null effect for verbal -lAr persists when verbal plural
morphology is the only potential attractor in the experiment.

\section{Experiment 2: Replication}\label{experiment-2-replication}

\subsection{Participants}\label{participants-1}

We recruited 80 undergraduate students to participate in the experiment
in exchange for course credit. All participants were native Turkish
speakers, with an average age of 21 (range: 18 -- 31). The experiment
was carried out following the principles of the Declaration of Helsinki
and the regulations concerning research ethics at Bogazici University.
All participants provided informed consent before their participation
and their identities were completely anonymised.

\subsection{Materials and Procedure}\label{materials-and-procedure}

Experiment 2 used only the verbal attractor conditions from Experiment
1. The procedure was identical to that of Experiment 1.

\subsection{Analysis and Results}\label{analysis-and-results-1}

Participants showed high accuracy in both grammatical (M = 0.94, CI =
{[}0.92,0.95{]}) and ungrammatical filler sentences (M = 0.08, CI =
{[}0.07,0.1{]}), indicating that they understood the task and performed
it reliably.

Figure~\ref{fig-exp1-condition-means} presents the overall means and
credible intervals for `yes' responses across experimental conditions.
As shown, ungrammatical sentences with plural attractors were rated as
acceptable as their counterparts with singular attractors (M = 0.94 and
0.95, CI = {[}0.93, 0.96{]} and {[}0.93, 0.97{]} for singular and plural
attractors, respectively).

On the other hand, accuracy in grammatical conditions was modulated by
the number of the attractor in an unexpected way. Participants rated
grammatical sentences with singular attractors as grammatical less often
(M = 0.92, CI = {[}0.9,0.94{]}) compared to their counterpars with
plural attractors (M = 0.95, CI = {[}0.93,0.96{]}).

\begin{figure}

\centering{

\pandocbounded{\includegraphics[keepaspectratio]{paper_files/figure-pdf/fig-exp1-condition-means-1.pdf}}

}

\caption{\label{fig-exp1-condition-means}Mean proportion of `acceptable'
responses by grammaticality and attractor number. Error bars show 95\%
Clopper--Pearson confidence intervals.}

\end{figure}%

These descriptive trends were confirmed by our Bayesian mixed-effects
models implemented in brms, assuming a Bernoulli logit link. The model
was fitted to the binary \emph{yes/no} responses and included fixed
effects for Grammaticality and Attractor Number and their interaction,
and random intercepts and slopes for both subjects and items.

Posterior estimates are summarized in
Figure~\ref{fig-exp1-fixed-effects}. The model revealed a positive
effect of grammaticality (\(\beta\) = 5.92 {[}5.40, 6.46{]}, P(\(\beta\)
\textgreater{} 1.00)), but no reliable main effect of attractor number
(\(\beta\) = 0.16 {[}-0.19, 0.51{]}, P(\(\beta\) \textgreater{} 0.81)).
On the other hand, there was a small but positive interaction (\(\beta\)
= 0.66 {[}-0.03, 1.37{]}, P(\(\beta\) \textgreater{} 0.97)). To clarify
the effects' presence in grammaticals only, we fitted two more models
that is fitted to the subset of the data. While the model fitted to
grammatical conditions only showed an effect of attractor number
(\(\beta\) = 0.51 {[}0.06, 0.98{]}, P(\(\beta\) \textgreater{} 0.99)),
the model fitted to ungrammatical conditions did not provide evidence
for the effect of number manipulation (\(\beta\) = -0.04 {[}-0.45,
0.37{]}, P(\(\beta\) \textgreater{} 0.99)). These results suggest that
the presence of a plural attractor did not increase the acceptability of
ungrammatical sentences, nor was this relationship modulated by
grammaticality.

\begin{figure}

\centering{

\pandocbounded{\includegraphics[keepaspectratio]{paper_files/figure-pdf/fig-exp1-fixed-effects-1.pdf}}

}

\caption{\label{fig-exp1-fixed-effects}Posterior means and 95\% credible
intervals for fixed effects in the two Bayesian models. The x-axis shows
the posterior mean (log-odds scale). The blue intervals correspond to
the model in which a positive interaction was assumed, and the orange
intervals to the model in which it was not.}

\end{figure}%

\subsection{Discussion}\label{discussion-1}

Experiment 1 tested whether

Unexpectedly, grammatical sentences with singular attractors were judged
less acceptable than those with plural attractors. This effect is
unlikely to reflect agreement attraction, since it arises in the
opposite direction. One possibility is that it results from an
interaction between plausibility and referential availability. The
plural morpheme can license a more general interpretation by allowing an
arbitrary or unspecific reference, whereas the singular reduced relative
clause more strongly invites a specific referent, which may be less
accessible in the context of the task. In other words, plural morphology
may facilitate an \emph{arbitrary PRO} interpretation of the embedded
clause, in which the understood subject of the relative clause is not
controlled by any overt antecedent and has a generic or impersonal
reference. A similar effect can be seen in English sentences like `Just
to sit there should be forbidden.' Here, the subject of the infinitival
clause has arbitrary reference. We do not pursue this explanation
further, as it falls outside the scope of the present paper.

\section{General Discussion}\label{general-discussion}

In two high-powered speeded acceptability judgment experiments, we
tested whether pure phonological overlap between agreement morphemes can
elicit agreement attraction. Our goal was to evaluate previous accounts
that attribute attraction to accidental or non-accidental syncretism
between forms that can serve as agreement controllers. Turkish provides
a useful test case because the plural suffix -lAr appears both on verbs
and on nouns, but only nominal -lAr can control agreement. If
phonological overlap alone can activate controller-relevant cues, then
plural-marked verbs embedded in reduced relative clauses should induce
attraction effects even though they cannot syntactically control
agreement.

Across both experiments, we found that Turkish attraction is determined
by being a potential controller rather than merely resembling one.
Participants did not produce or endorse attraction errors in sentences
containing verbal attractors, and this absence of attraction persisted
even when the same participants showed robust attraction with nominal
attractors in the same session.

These results indicate that attraction depends on abstract feature
overlap with potential controllers, not on surface-form similarity. This
pattern converges with prior results in English and Turkish that failed
to find attraction for pseudoplural or phonologically plural forms
\citep{BockEberhard1993, HaskellMacDonald2003, NicolEtAl:2016}, and
stands in contrast to findings from Russian \citep{Slioussar2018}.

In \citet{Slioussar2018}, genitive-marked singular nouns that were
homophonous with nominative plurals elicited greater attraction effects
than their genitive-plural counterparts. This is striking because the
relevant nouns lacked a plural feature that could percolate or serve as
a retrieval cue. The effect was therefore interpreted as evidence that
comprehenders can use phonological form to activate abstract agreement
features. However, it is important to note that the evidence for
phonological attraction in Russian rests on a small empirical base. The
production and comprehension experiments in \citep{Slioussar2018}
included only 32 participants each, and the attraction effects were
derived from a small number of error trials (13 in production and 18 in
comprehension). Given the low number of critical observations, such
effects are vulnerable to sampling variability and may not generalize
beyond that dataset.

The high-powered Turkish results challenge that interpretation. Despite
identical surface overlap between verbal and nominal plural morphology,
phonological similarity alone did not yield attraction. This
cross-linguistic contrast suggests that form-based activation of
agreement features is not a universal property of the parsing system
but, at best, depends on language-specific mappings between morphology
and syntactic function \citep{DillonKeshev2024}.

A more plausible account is that attraction is modulated by the
availability of morphosyntactic features that can signal controllerhood.
Syncretism contributes to attraction only when one of the syncretic
forms can legitimately control agreement or share features with the
target. In other words, it is not form overlap per se, but feature
ambiguity that matters. This interpretation aligns with cross-linguistic
findings showing that attraction is strongest when the attractor bears
case or number morphology that is sometimes associated with subjects or
agreement controllers
\citep{LagoEtAl2019, BhatiaDillon2022, BleotuDillon2024, HartsuikerEtAl2003}.
Earlier formulations of these models left open whether `looking like' a
controller or `being able to be' a controller was critical. The present
results favor the latter: only morphologically licensed controllers
engage in attraction.

\section*{References}\label{references}
\addcontentsline{toc}{section}{References}

\newcommand{\doi}[1]{\href{http://dx.doi.org/#1}{http://dx.doi.org/#1}}
\begingroup
\raggedright
\singlespacing

\renewcommand{\bibsection}{}
\bibliography{bibliography.bib}

\endgroup





\end{document}
