% Options for packages loaded elsewhere
% Options for packages loaded elsewhere
\PassOptionsToPackage{unicode}{hyperref}
\PassOptionsToPackage{hyphens}{url}
\PassOptionsToPackage{dvipsnames,svgnames,x11names}{xcolor}
%
\documentclass[
  10pt,
  letterpaper,
]{article}
\usepackage{xcolor}
\usepackage[margin=1in]{geometry}
\usepackage{amsmath,amssymb}
\setcounter{secnumdepth}{5}
\usepackage{iftex}
\ifPDFTeX
  \usepackage[T1]{fontenc}
  \usepackage[utf8]{inputenc}
  \usepackage{textcomp} % provide euro and other symbols
\else % if luatex or xetex
  \usepackage{unicode-math} % this also loads fontspec
  \defaultfontfeatures{Scale=MatchLowercase}
  \defaultfontfeatures[\rmfamily]{Ligatures=TeX,Scale=1}
\fi
\usepackage{lmodern}
\ifPDFTeX\else
  % xetex/luatex font selection
  \setmainfont[]{Charis SIL}
\fi
% Use upquote if available, for straight quotes in verbatim environments
\IfFileExists{upquote.sty}{\usepackage{upquote}}{}
\IfFileExists{microtype.sty}{% use microtype if available
  \usepackage[]{microtype}
  \UseMicrotypeSet[protrusion]{basicmath} % disable protrusion for tt fonts
}{}
\usepackage{setspace}
\makeatletter
\@ifundefined{KOMAClassName}{% if non-KOMA class
  \IfFileExists{parskip.sty}{%
    \usepackage{parskip}
  }{% else
    \setlength{\parindent}{0pt}
    \setlength{\parskip}{6pt plus 2pt minus 1pt}}
}{% if KOMA class
  \KOMAoptions{parskip=half}}
\makeatother
% Make \paragraph and \subparagraph free-standing
\makeatletter
\ifx\paragraph\undefined\else
  \let\oldparagraph\paragraph
  \renewcommand{\paragraph}{
    \@ifstar
      \xxxParagraphStar
      \xxxParagraphNoStar
  }
  \newcommand{\xxxParagraphStar}[1]{\oldparagraph*{#1}\mbox{}}
  \newcommand{\xxxParagraphNoStar}[1]{\oldparagraph{#1}\mbox{}}
\fi
\ifx\subparagraph\undefined\else
  \let\oldsubparagraph\subparagraph
  \renewcommand{\subparagraph}{
    \@ifstar
      \xxxSubParagraphStar
      \xxxSubParagraphNoStar
  }
  \newcommand{\xxxSubParagraphStar}[1]{\oldsubparagraph*{#1}\mbox{}}
  \newcommand{\xxxSubParagraphNoStar}[1]{\oldsubparagraph{#1}\mbox{}}
\fi
\makeatother


\usepackage{longtable,booktabs,array}
\usepackage{calc} % for calculating minipage widths
% Correct order of tables after \paragraph or \subparagraph
\usepackage{etoolbox}
\makeatletter
\patchcmd\longtable{\par}{\if@noskipsec\mbox{}\fi\par}{}{}
\makeatother
% Allow footnotes in longtable head/foot
\IfFileExists{footnotehyper.sty}{\usepackage{footnotehyper}}{\usepackage{footnote}}
\makesavenoteenv{longtable}
\usepackage{graphicx}
\makeatletter
\newsavebox\pandoc@box
\newcommand*\pandocbounded[1]{% scales image to fit in text height/width
  \sbox\pandoc@box{#1}%
  \Gscale@div\@tempa{\textheight}{\dimexpr\ht\pandoc@box+\dp\pandoc@box\relax}%
  \Gscale@div\@tempb{\linewidth}{\wd\pandoc@box}%
  \ifdim\@tempb\p@<\@tempa\p@\let\@tempa\@tempb\fi% select the smaller of both
  \ifdim\@tempa\p@<\p@\scalebox{\@tempa}{\usebox\pandoc@box}%
  \else\usebox{\pandoc@box}%
  \fi%
}
% Set default figure placement to htbp
\def\fps@figure{htbp}
\makeatother





\setlength{\emergencystretch}{3em} % prevent overfull lines

\providecommand{\tightlist}{%
  \setlength{\itemsep}{0pt}\setlength{\parskip}{0pt}}



 
\usepackage[]{natbib}
\bibliographystyle{apalike}


% Linguistics examples
\usepackage{gb4e}
\noautomath

% Glossing
\usepackage{leipzig}
\usepackage{typgloss}

% Tables
\usepackage{makecell}
\usepackage{booktabs}
\usepackage{multirow}
\usepackage{longtable}

% Line numbers for review
\usepackage{lineno}
\linenumbers

\usepackage{booktabs}
\usepackage{longtable}
\usepackage{array}
\usepackage{multirow}
\usepackage{wrapfig}
\usepackage{float}
\usepackage{colortbl}
\usepackage{pdflscape}
\usepackage{tabu}
\usepackage{threeparttable}
\usepackage{threeparttablex}
\usepackage[normalem]{ulem}
\usepackage{makecell}
\usepackage{xcolor}
\makeatletter
\@ifpackageloaded{caption}{}{\usepackage{caption}}
\AtBeginDocument{%
\ifdefined\contentsname
  \renewcommand*\contentsname{Table of contents}
\else
  \newcommand\contentsname{Table of contents}
\fi
\ifdefined\listfigurename
  \renewcommand*\listfigurename{List of Figures}
\else
  \newcommand\listfigurename{List of Figures}
\fi
\ifdefined\listtablename
  \renewcommand*\listtablename{List of Tables}
\else
  \newcommand\listtablename{List of Tables}
\fi
\ifdefined\figurename
  \renewcommand*\figurename{Figure}
\else
  \newcommand\figurename{Figure}
\fi
\ifdefined\tablename
  \renewcommand*\tablename{Table}
\else
  \newcommand\tablename{Table}
\fi
}
\@ifpackageloaded{float}{}{\usepackage{float}}
\floatstyle{ruled}
\@ifundefined{c@chapter}{\newfloat{codelisting}{h}{lop}}{\newfloat{codelisting}{h}{lop}[chapter]}
\floatname{codelisting}{Listing}
\newcommand*\listoflistings{\listof{codelisting}{List of Listings}}
\makeatother
\makeatletter
\makeatother
\makeatletter
\@ifpackageloaded{caption}{}{\usepackage{caption}}
\@ifpackageloaded{subcaption}{}{\usepackage{subcaption}}
\makeatother
\usepackage{bookmark}
\IfFileExists{xurl.sty}{\usepackage{xurl}}{} % add URL line breaks if available
\urlstyle{same}
\hypersetup{
  pdftitle={(In)sensitivity to surface-level heuristics: A case from Turkish verbal attractors},
  pdfauthor={Utku Turk},
  pdfkeywords={form-sensitivity, memory, agreement
attraction, linguistic illusions, sentence processing},
  colorlinks=true,
  linkcolor={blue},
  filecolor={Maroon},
  citecolor={Blue},
  urlcolor={Blue},
  pdfcreator={LaTeX via pandoc}}


\title{(In)sensitivity to surface-level heuristics: A case from Turkish
verbal attractors}
\author{Utku Turk}
\date{2026-02-24}
\begin{document}
\maketitle
\begin{abstract}
Linguistic illusion literature debates what information accesses memory
representations. Prior work tests whether structural, semantic, or
discourse cues guide subject-verb dependencies; however, it remains
unclear whether native speakers rely on surface level heuristics, such
as phonological information during dependency resolution. Traditionally,
accidental phonological resemblance to plural ending (e.g., the /s/ in
\emph{cruise}) does not induce erroneous agreement in English, whereas
resemblance correlating with controllerhood amplifies attraction across
varies languages. Contradicting this generalization, Slioussar (2018)
proposed that accidental phonological resemblance can mediate memory
search for Russian subjects. Given the theoretical importance of this
proposal and the lack of comparable effects in other languages such as
Czech, we propose re-interpret previous findings under the light of a
recently growing literature of association with being a possible
controller. We test whether phonological overlap or association with
controllerhood elicits erroneous agreement in Turkish. Turkish provides
a critical test: both verbal and nominal elements can surface as
subjects and the plural morpheme \emph{-lAr} marks number in both of
them, but only nominal plural \emph{-lAr} controls verbal agreement. Two
speeded acceptability studies show no attraction from plural-marked
verbs (N = 80; N = 95) but robust attraction from genitive plural nouns.
We report a first-of-its-kind dissociation under minimal manipulation:
verbal attractors that can surface as subjects yet cannot control
agreement do not induce attraction, whereas genitive plural
nouns---which can be subjects and control in other environments---do.
This pattern constrains retrieval processes by tying attraction to
abstract controller features rather than surface phonology.
\end{abstract}


\setstretch{1}
\section{Introduction}\label{introduction}

Human sentence processing draws both on abstract grammatical features
and heuristics that exploit surface regularities, such as plausibility
\citep{SpeerClifton1998}, frequency \citep{LauEtAl2007}, and
task-specific factors
\citep{LauraMalsbug24, ArehalliWittenberg2021, HammerlyEtAl2019, LogacevVasishth2016}.
We focus on one such heuristic: over-reliance on surface form, evidenced
when phonological similarity between sentence constituents is observed
to modulate performance
\citep{AchesonMacDonald2011, KushEtAl2015, CopelandRadvansky2001, RastleDavis2008}.

A substantial body of work has shown that the parser and the production
system are sensitive not only to syntactic or semantic relations but
also to the surface form of words. These effects have been taken to
suggest that, under certain circumstances, speakers and comprehenders
rely on shallow or heuristic cues to complete dependencies.
\citet{AchesonMacDonald2011}, for example, found that participants
showed slower reading times when the subjects of the two embedded
clauses share phonological similarity (\emph{baker}-\emph{banker} in
\ref{baker} vs.~\emph{runner}-\emph{banker} in \ref{runner}). Moreover,
participants were less accurate in answering comprehension questions
with phonological overlap present. Related work in short-term memory and
word recognition shows similar effects---items that overlap
phonologically or morphologically are more confusable and more easily
retrieved (Copeland \& Radvansky, 2001; Rastle \& Davis, 2008).

\begin{exe}
\ex \label{baker} The baker that the banker sought bought the house.
\ex \label{runner} The baker that the banker sought bought the house.
\end{exe}

However, it is unresolved whether this heuristic penetrates dependency
resolution itself--including subject-verb agreement, pronoun resolution,
or the licensing of negative polarity items--beyond general effects on
reading ease and memory. A central question for understanding human
cognition is what information is encoded and later available in memory
during such dependency resolutions and how faithful these encodings are
to the input. Errors in subject-verb agreement have been treated as a
key domain for identifying mechanisms of linguistic representation and
retrieval \citep{BockMiller:1991, JAEGAR, SmithVasishth, PHILLIPS2011}.
Classic findings demonstrate systematic errors in establishing number
agreement between a verb and its agreement controller when an NP with a
different number (the attractor) interferes. Speakers produce sentences
like (\ref{og}) or misclassify them as acceptable
\citep{BockMiller:1991, PearlmutterGarnseyBock:1999}.

\begin{exe}
\ex[*]{\label{og} The player on the courts are tired from a long-game.}
\ex[]{\label{og-mod} The players on the courts are tired from a long-game.}
\end{exe}

Some accounts argue that detailed analyses are not always maintained
when heuristics suffice, creating the opportunity for surface
regularities to affect judgments
\citep{FerreiraEtAl2002, FUTRELLETAL2020}. For example,
lossy-compression-style approaches assume that comprehenders maintain
only an imperfect representation of the linguistic input, and that the
parser relies on statistical regularities within the language to fill in
the gaps \citep{FUTRELLETAL2020}. In this model, the input as in
(\ref{og}) would be altered to a more statistically predictable form
such as (\ref{og-mod}), which would be accepted as grammatical. On the
other hand, many rational accounts of sentence processing argue that
comprehenders maintain detailed and faithful representations of the
input \citep{BockMiller:1991, LV05}. More specifically, cue-based
retrieval approaches hold that constituents are stored with detailed
abstract features and later accessed by matching retrieval cues, and
that erroneous parses can occur when features conflict or interfere. In
such accounts, the parser would maintain a detailed representation of
the singular subject \emph{player} and the plural attractor
\emph{courts}, and the error arises because the retrieval cues for
agreement (+PL, +CONTROLLER) match both the singular subject and the
plural attractor, leading to interference. However, it remains open
whether phonological codes are used as such cues during syntactic
dependency building \citep{LV05}.

Despite much research on what factors modulate agreement errors, the
role of phonology remains unclear. The few studies that bear directly on
subject-verb agreement exhibit contradictory findings
\citep{BockEberhard1993, Slioussar2018, LacinaChromy2022}. Pseudoplural
attractors whose final phone matches the plural suffix and
string-ambiguous with other nouns (e.g.~\emph{cruise} vs.~\emph{crews})
do not increase agreement errors in production \citep{BockEberhard1993}.
Phonological overlap effects have been observed in other cases, but many
of them involve additional shared morphological features such as case
ambiguity with the controller in the sentence
\citep{HartsuikerEtAl2003, LagoEtAl2019, LacinaChromy2022}, although not
all \citep{Slioussar2018}.

This raises the possibility that surface form affects the formation of
agreement dependencies not directly through the use of number form as a
retrieval cue, but indirectly, when the surface form is one that is more
likely to be realized on agreement controllers.

We test this hypothesis by utilizing the surface-form overlap between
the verbal and nominal morphological reflexes of agreement in Turkish.
Turkish uses the same surface suffix, \emph{-lAr}, for plural marking on
nouns and for plural agreement on finite verbs. Crucially, strings
bearing verbal \emph{-lAr} can occur in subject position, yet they never
control finite clause agreement; only nominal plurals do. These
properties allow us to test whether form overlap is sufficient to drive
attraction, or if the attractor must also be a possible controller (true
of nouns but not verbs). Across two high-powered speeded acceptability
experiments in Turkish we find that plural marking on an embedded verbal
attractor does not increase acceptance of plural agreement on the matrix
verb; such effects are only observed when the plural marker appears on a
non-subject noun attractor. These results indicate that surface-form
overlap alone does not function as a retrieval cue for agreement in
Turkish. Dependency resolution relies on abstract features and
structural relations, with phonology influencing processing primarily
outside of retrieval.

In the rest of the introduciton, we review the role of certain surface
cues in agreement attraction, namely case syncretism and pure
phonological overlap. We then introduce the previous Turkish attraction
studies and brief sketch of the grammar that is relevant for our study.
Finally, we present the current study and its predictions.

\subsection{Background}\label{background}

It is reported that native speakers from 296 of out 378 languages
surveyed exhibit systematic agreement between the verb and another
constituent(s), such as subject, object, or both \citep{WALS}. However,
this agreement process is not always error-free. In their seminal work,
\citet{BockMiller:1991} demonstrated that participants systematically
produce erroneous verb forms (\emph{are}) when there is a nearby noun,
an attractor, that has a mismatching number as in (\ref{true-pl})
compared to their counterpart with singular attractor as in
(\ref{true-sg}). The effect of the number mismatching attractor,
agreement attraction, was also found to be robust in comprehension
\citep{NicolEtAl1997, PearlmutterGarnseyBock:1999} of such sentences in
various languages, including Arabic \citep{TuckerEtAl:2015}, Armenian
\citep{AvetisyanEtAl:2020}, Hindi \citep{BhatiaDillon2022}, Spanish
\citep{LagoEtAl2015}, Russian \citep{Slioussar2018}, and Turkish
\citep{LagoEtAl2019, TurkLogacev2024, Ulusoy2023}.

\begin{exe}
\ex \label{initial}
\begin{xlist}
    \ex \label{true-sg} {Singular Attractor} \\ The {player} on the {court} \ldots{}
    \ex \label{true-pl} {Plural Attractor} \\ The {player} on the {courts} \ldots{}
\end{xlist}
\end{exe}

Many studies have investigated the various syntactic and semantic
factors which make agreement errors more likely, which include
hierarchical distance
\citep{HatsuikerEtAl2001, NicolEtAl1997, Kaan2002}, linear distance
(Bock and Cutting \citeyearpar{BockCutting1992}; but see Pearlmutter
\citeyearpar{Pearlmutter2000} and Kwon and Strut
\citeyearpar{KwonStrut2019}), semantic interactions of nouns involved
\citep{Eberhard1999, ViglioccoEtAl95, HumphreysBock2005}, and syntactic
category of the phrase containing the attractor
\citep{BockMiller:1991, BockCutting1992}. One widely accepted set of
accounts that explained these errors are called retrieval based theories
\citep{LV05, WagersEtAl:2009, YADAV}. In these accounts, comprehenders
maintain faithful linguistic representations; errors arise because the
memory mechanisms used to identify the agreement controller mislead
them. Under this approach, phrases are encoded in content-addressable
memory as \emph{chunks}---bundles of features including number, gender,
and syntactic properties \citep{SmithVasishth2020}. Comprehenders
predict the number of the verb based on the noun phrases they process
while reading the previous noun phrases. In grammatical sentences with
singular verb agreement, the number prediction and the verb number
match, which causes no processing difficulty. In contrast, when
participants fail to find the predicted number morphology on the verb, a
memory-retrieval process is initiated. This process activates the search
for a chunk matching relevant cues for agreement controller.

\subsubsection{Surface Heuristics: case syncretism and phonological
overlap}\label{surface-heuristics-case-syncretism-and-phonological-overlap}

What is the characteristics of cues which are found useful to be
encoded? One line of work manipulated overt case marking on attractors,
i.e.~syncreticsm, to test whether morphophonological case is used for
dependency resolution. Two grammatical forms are said to be syncretic if
they are realized with the same overt morphology despite bearing
different syntactic and semantic features. E.g., most noun phrases in
English -- such as \textit{the cabinet} -- are syncretic between
nominative and accusative case marking. An exception are some pronouns
which differ in their nominative and accusative forms, as with the first
person pronoun \textit{I} (nominative) vs.~\textit{me} (accusative).

This effect of case syncretism was tested in various languages by
manipulating the overt case marking of controllers or attractors,
reasoning that surface ambiguity could enhance competition during
retrieval or interfere in production. For example,
\citet{HartsuikerEtAl2003} in a preamble completion task experiment used
the overlap between accusative and nominative forms of feminine
determiners in German and compared these ambiguous forms to
distinctively marked dative forms. Participants produced more agreement
errors when the preambles contained two noun phrases whose determiners
were not distinctively marked, as in (\ref{ger-amb}), compared to cases
where the attractor could be distinguished by form alone, as in
(\ref{ger-dist}). Crucially, this additive effect was limited to
feminine nouns, the only gender showing nominative--accusative
syncretism in plural forms while other nouns showed the base effect of
plural.

\begin{exe}
\ex \label{ger}
\begin{xlist}
\ex \label{ger-amb}
\gll Die Stellungnahme gegen die Demonstration-en\\
the.F.NOM.SG position against the.F.ACC.PL demonstration-PL\\
\glt `The position against the demonstrations'
\ex \label{ger-dist}
\gll Die Stellungnahme zu den Demonstration-en\\
the.F.NOM.SG position on the.F.DAT.PL demonstration-PL\\
\glt `The position on the demonstrations'
\end{xlist}
\end{exe}

Parallel results were found in comprehension studies in Czech.
\citet{ChromyEtAl2023} conducted self-paced reading experiments
manipulating the number of the attractor and the number of the verb
across varies syntactic configurations. They found that the attraction
effects, i.e.~faster reading times at the ungrammatical verb when the
attractor is plural, were only observed when the attractor was bearing a
case syncretic with the nominative case as in (\ref{czech-amb}), while
finding no attraction effects in other experiments where the attractor
was unambiguously marked. In a follow up experiment,
\citet{LacinaEtAl2025} found similar effects in gender agreement within
Czech, based on earlier Slovak production findings
\citep{BadeckerKuminiak2007}. They found a clear gender attraction
effect, i.e.~faster reading times at the ungrammatical verb when the
attractor is of the same gender with the verb, but the head noun was
not. More importantly, this effect was only present in cases where the
attractor was syncretic in case-marking with the nominative case, which
is the case of the agreement controller, but not when the attractor was
unambiguously marked with a different case.

\begin{exe}
\ex[*]{\label{czech-amb}
\gll Slo\v{z}k-a pro archiv\'a\v{r}k-y nejsp\'is bud-ou zahrnovat ve\v{s}kere n\'alezy.\\
file-NOM.SG for archiever-ACC.PL=NOM.PL probably will-NOM.PL include all findings\\
\glt `A file for archievers will probably include all findings.'
}
\end{exe}

Similarly, \citet{Slioussar2018} found the effects of syncretism in
Russian in both production, self-paced readings, and acceptability
judgments. She compared sentences with genitive plural attractors, which
are unambiguously marked, to sentences with accusative plural
attractors, which are syncretic with nominative plural forms, while
manipulating other factors such as the number of the attractor, the
grammaticality of the sentence, and the presence of a singular subject.
She found that sentences with accusative plural attractors yielded more
plural completions, faster reading times at the plural verb and higher
rates of acceptability compared to the sentences with unambiguous
genitive plural attractors.

However, results from other case-marking languages are mixed. For
instance, \citet{FrankEtAl2010} used French and compared the
unambiguously accusative marked attractors to NPs with no overt case
marking. They showed that when unambiguous marking increased the
attraction effects substantially, contrary to the predictions of cue
based retrieval. \citet{AvetisyanEtAl:2020} observed that unambiguous
case in Armenian modulated neither reading times nor error rates.
Conversely, \citet{LacinaEtAl2025} found that attraction in Czech
surfaced only when case morphology was ambiguous. These findings suggest
that distinct case morphology is insufficient to predict interference,
implicating language-specific distributions or heuristic processing.

The studies discussed above tested the effects of case syncretism, which
is a morphophonological overlap that also correlates with the
possibility of being a controller. In these cases, while the case of the
attractor is ambiguous its number is not. Take the English word
\emph{cabinets} as an example. It is syncretic between nominative and
accusative case, meaning that its surface form would not change
depending on the syntactic case is assigned to it. However, it is not
syncretic in number, as the plural form \emph{cabinets} is distinct from
the singular form \emph{cabinet}, and this difference would surface as
reflex on grammaticality in syntactic and semantic configurations where
a certain number is expected, such as \emph{*few cabinet} and \emph{*a
cabinets}.

A second line of work related to surface cues tests a case of accidental
phonological overlap that does not itself change the relevant cues.
\citet{BockEberhard1993} tested whether attractors that only sound
plural, pseudoplural singular attractors such as \emph{cruise} as in
(\ref{pseudoplural}), increase agreement errors compared to true plural
nouns, such as \emph{crews} in (\ref{true-pl}). They reasoned that if
participants rely on phonological cues rather than abstract number
features, words ending with plural-like sounds (/s/ or /z/) should
behave like true plurals. In their preamble completion study, they found
that pseudoplural attractors did not induce agreement errors, which
argues against a purely phonology-driven account of attraction in
English.

\begin{exe}
\ex \label{pseudoplural} The {player} on the {cruise} \ldots{}
\ex \label{not-pseudoplural} The {player} on the {crews} \ldots{}
\end{exe}   

In contrast, \citet{Slioussar2018} reported a contribution of
surface-form overlap to agreement in Russian. Recall that she compared
genitive and accusative marked attractors. In Russian, a subset of
genitive singular nouns (\ref{russian-gen-sg}) is homophonous with
nominative plural forms, while genitive plural forms
(\ref{russian-gen-pl}) are not ambiguous in this way. In the experiments
we previously mentioned, she found that sentences with genitive singular
attractors whose form overlaps with nominative plural yielded more
plural completions, faster reading times at the plural verb and higher
rates of acceptability compared to the sentences with unambiguous
genitive plural attractors. She took her results as evidence for a
retrieval process in which the search for a controller is mediated
through phonological form and relevant features like +NOM and +PL can be
activated by simply a phonological overlap.

\begin{exe}
\ex \label{russian-gen-sg}
    \gll Komnata dlja ve\v{c}erinki byli ...\\
    room.\textsc{nom}.\textsc{sg} for party.\textsc{gen}.\textsc{sg}=\textsc{nom}.\textsc{pl} were\\
    \glt`The room for parties/party were ...'
\ex \label{russian-gen-pl} 
    \gll Komnata dlja ve\v{c}erinok byli ...\\
    room.\textsc{nom}.\textsc{sg} for party.\textsc{gen}.\textsc{pl}$\neq$\textsc{nom}.\textsc{pl} were\\
    \glt `The room for parties/party were ...'
\end{exe}

However, another Slavic language Czech which shows the same ambiguity
between the genitive singular and nominative plural forms was found to
not show attraction effects by simple phonological overlap
\citep{LacinaChromy2022}. These mixed findings in case-syncretism
literature, English pseudoplural, and a failure to replicate in another
Slavic language cast a shadow on phonological modulation explanation.

\subsubsection{Accounting for syncretism and
phonological-overlap}\label{accounting-for-syncretism-and-phonological-overlap}

To the best of our knowledge, there is no attempt of explanation within
Rational Accounts or Marking and Morphing accounts to explain the
variety of findings in the literature regarding surface cues in
agreement attraction. Below we outline existing models of the syncretism
effects under a cue-based approach, and then how other account would
approach these findings.

In canonical cue-based retrieval accounts, a dependency is formed via a
content-addressable search for a relevant \emph{chunk}, i.e.~bundles of
features for a given lexical item, that matches the retrieval cues
provided by the verb \citep{LV05}. When a verb encountered, a search is
triggered for a chunk that matches the necessary features for agreement.
In the case of English subject-verb agreement, the verb would provide
cues such as +NOM and +PL, and the search would activate chunks that
match these cues. If there is a singular subject and a plural attractor,
both of which match the +NOM cue but only the plural attractor matches
the +PL cue, then the retrieval process can be misled by the attractor,
leading to agreement errors.\footnote{\citet{LV05} argue that the
  retrieval cues for case is not determined by the syntactic position or
  the abstract case features, but rather by the surface form of the noun
  phrase.} In this account, the case syncretism is expected to lead to
difficulty in correctly determining the head due to the increased
competition between chunks that shares relevant cues. On the other hand,
when the attractor is not case-syncretic, the partial match between the
retrieval cues and the attractor would be weaker, leading to less
attraction effects \citep{SmithVasishth, YADAV}.

\citet{Slioussar2018} situated her findings within the cue-based
retrieval accounts. She argued that the phonological overlap between the
genitive singular and nominative plural forms can activate relevant
features like +NOM and +PL in addition to +GEN and +SG, which would
increase the likelihood of retrieving the attractor as a controller,
leading to more agreement errors in cases where the attractor only
matches with the +GEN and +PL cues (\ref{russian-gen-sg}). To the best
of our knowledge, this is the only account that directly exploits the
co-activation of chunks through phonological overlap to explain the
effects of syncretism. However, Slioussar's (2018) mechanism aligns
closely with foundational principles of word recognition and working
memory. Models of continuous speech parsing
\citep[e.g.,][\citet{Norris1994}]{McClellandElman1986} have long
established that phonological string overlap automatically triggers the
transient activation of embedded or competing lexical chunks
\citep{Shillcock1990}. Furthermore, the broader working memory
literature demonstrates that phonological similarity creates significant
interference during retrieval
\citep{Baddeley1966, Conrad1964, AchesonMacDonald2009}

However, the same surface-form overlap did not give rise to attraction
effects in Czech, another Slavic language \citet{LacinaChromy2022}.
These mixed findings in case-syncretism literature, English
pseudoplural, and a failure to replicate in another Slavic language cast
a shadow on phonological modulation explanation.

An alternative account posits that attraction errors arise not from
phonological co-activation of competing parses, but from the use of
language-general statistical heuristics. Under this view, comprehenders
probabilistically associate certain surface cues---word order, case
syncretism, or the presence of certain morphemes---with controllerhood,
the property of being a possible agreement controller. In cases of
syncretism, then, certain noun phrases might carry an increased
association with controllerhood due to the distribution of such forms in
the language. For example, \citet{LagoEtAl2019} argue that Turkish
speakers retrieve genitive-marked attractors as controllers because
genitive case controls agreement in embedded clauses, even though it
cannot do so in matrix clauses. The syncretism between the nominal
modifier and the embedded subject is thus phonological rather than
functional, and attraction arises because Turkish speakers associate
genitive-marked nominals with being controllers.

Converging evidence for this sensitivity to ``looking like a
controller'' comes from Romanian and Hindi
\citep{BhatiaDillon2022, BleotuDillon2024}. \citet{BleotuDillon2024}
found that Romanian attractors induced agreement errors only when they
surfaced with a determiner, as opposed to bare forms. Since only nouns
bearing a determiner can control agreement in Romanian, they argue that
participants associate the presence of a determiner with controllerhood.
In Hindi, \citet{BhatiaDillon2022} found that plural-marked attractors
were erroneously retrieved as controllers only when the attractor also
served as an agreement controller within the embedded
clause---independent of whether its syntactic role was that of an object
or a subject. They argue that participants track controllerhood within a
sentence rather than relying on language-general distributional
statistics alone. Nevertheless, this finding also demonstrates that
agreement processes are sensitive to the abstract feature of being a
controller.

Further evidence comes from English. In a series of six experiments,
\citet{SchlueterEtAl2018} showed that the coordinator \emph{and} when
coordinating two singular noun phrases induces attraction even in the
absence of overt plural morphology \emph{-s}, because they argue
\emph{and} is statistically associated with plurality. Crucially, not
only conjoined singular noun phrases lacking overt plural morphology are
good candidates for attraction, but the coordinator \emph{and} alone can
induce the effect when it conjoins two adjectives modifying a singular
noun (e.g., \emph{the slogan about the loyal and caring husband}). They
argue that participants exploit the statistical association between
\emph{and} and plurality, which leads them to accept ungrammatical
sentences containing a plural auxiliary. Taken together, these
explanations across languages and structures suggest that the cue--chunk
match is not strictly categorical but can be influenced by statistical
associations within a language \citep{EngelmannEtAl2019}. Importantly,
the Hindi findings indicate that the human parser is sensitive to the
abstract notion of being a controller, and not merely to
language-general co-occurrence statistics.

A similar account extends to Russian. While genitive-marked nouns can
serve as subjects in negative inversion constructions, they do not
control verbal agreement in these contexts. Crucially, however, they
remain active controllers within the noun phrase, triggering number or
gender marking on modifiers (e.g., surfacing as feminine \emph{ni odnoy}
with a feminine head, contrasting with masculine \emph{ni odnogo} in
\ref{russian}) \citep{Babby2001, ParteeBorschev2004}. In contrast, Czech
does not allow genitive subjects, and thus not license these controller
properties in subject positions.

\begin{exe}
\ex \label{russian}
    \gll ..., tam ne rabotaet ni odnogo in\v{z}enera.\\
    ..., there NEG works not one.M.SG.GEN engineer.M.SG.GEN\\
    \glt `..., there hasn't been a single engineer working there.'
\end{exe}

\subsubsection{Sketch of Turkish and Attraction in
Turkish}\label{sketch-of-turkish-and-attraction-in-turkish}

Turkish offers a useful test case because genitive-marked nominals can
carry controller-like cues in some structures, while verbal agreement is
morphologically rich and overt. In genitive-possessive NPs (roughly
analogous to English Saxon genitives), the possessor bears genitive case
and the head noun bears possessive morphology.

Using this construction, \citet{LagoEtAl2019} reported robust
attraction: participants accepted ungrammatical plural agreement more
often when the possessor was plural (\ref{lago-tr-pl}) than when it was
singular (\ref{lago-tr-sg}).

\begin{exe}
\ex \label{lago-tr-pl}
    \gll Teknisyen-ler-in e\u{g}itmen-i  ola\u{g}an\"ust\"u h{\i}zl{\i} ko\c{s}-tu-lar.\\
    technician-PL-GEN instructor-POSS extraordinary fast run-PST-PL\\
    \glt `The technicians' instructor ran$_{PL}$ extraordinarily fast.'
\ex \label{lago-tr-sg}
    \gll Teknisyen-in e\u{g}itmen-i  ola\u{g}an\"ust\"u h{\i}zl{\i} ko\c{s}-tu-lar.\\
    technician-GEN instructor-POSS extraordinary fast run-PST-PL\\
    \glt `The technician's instructor ran$_{PL}$ extraordinarily fast.'
\end{exe}

\citet{TurkLogacev2024} asked whether this attraction might be partly
driven by a local parsing ambiguity in the original materials. In
\citet{LagoEtAl2019}, many head nouns were consonant-final, so the head
suffix \emph{-i} was syncretic between 3sg possessive and accusative.
This creates a potential misparse in which the genitive possessor is
temporarily treated as a clause-level subject and the head as an object,
which could artificially increase the possessor's controller-like
status.

To test this ambiguity account, \citet{TurkLogacev2024} used vowel-final
heads that disambiguate the two morphemes: 3sg possessive surfaces as
\emph{-si}, whereas accusative surfaces as \emph{-yi}. The logic was
straightforward: if attraction in Turkish genitive-possessive NPs mainly
comes from this local ambiguity, then attraction should be substantially
reduced when the head morphology is unambiguous.

The attraction effect persisted. Plural genitive possessors still
increased ungrammatical acceptability, and the magnitude was comparable
to \citet{LagoEtAl2019}. This result argues against an ambiguity-only
explanation and suggests that genitive-marked possessors carry
controller-relevant cues independently of that local form overlap.

\citet{Ulusoy2023} extended this literature to configurations where the
attractor is not in the same phrase as the controller. Using
matrix-clause subjects as attractors for embedded verbal agreement
(\ref{ulusoy}), she found more errors with plural than singular
attractors. At the same time, ungrammatical acceptance was relatively
high in both conditions, despite cross-linguistic evidence that such
structural separation usually reduces attraction
\citep{BockCutting1992, FranckEtAl2002}. Taken together, Turkish studies
establish reliable attraction with plural genitive cues, but they leave
open which cues are doing the work: surface form, controllerhood, or
both.

\begin{exe}
\ex[*]{\label{ulusoy}
\gll K\"ut\"uphaneci-ler {[~\c{c}al{\i}\c{s}kan} \"o\u{g}renci-nin iste-dik-ler-i] kitab-{\i} \c{s}imdi bul-du-lar.\\
librarian-PL hardworking student.SG-GEN want-NMLZ-3SG-POSS book-ACC now find-PST-3PL\\
\glt `The librarians found the book that the hardworking student wanted now.'}

\end{exe}

\subsection{This study}\label{this-study}

Motivated by these alternative accounts and conflicting findings
\citep{BockEberhard1993, LacinaChromy2022} along with the theoretical
importance of such proposal, we test the phonological modulation
hypothesis in two high-powered experiments: whether a syntactically
ineligible controller, but still a possible subject, can induce
attraction solely through morphophonological overlap matching the
agreement suffix in form and semantics. To this end we capitalize on the
shared surface form of verbal and nominal plural marking (\emph{-lAr})
in Turkish to target this question. We use reduced relative clauses
(RRCs) where the plural-marked verb appears as the attractor
(\ref{rrc-intro}). Crucially, this \emph{-lAr} syncretism is not
feature-ambiguous; it is a form-only overlap lacking the possibility of
being a potential controller. Even when a headless RRC alone surfaces as
a subject, it cannot control agreement (\ref{rrc-subject}).

\begin{exe}
\ex \label{rrc-intro}
\gll Gör-dük-ler-i çocuk koş-tu-(*lar).\\
go-NMLZ-PL-POSS kid[NOM] run-PST-(*PL)\\
\glt `The kid that (they) saw ran.'
\ex \label{rrc-subject}
\gll Gör-dük-ler-i koş-tu-(*lar).\\
go-NMLZ-PL-POSS run-PST-(*PL)\\
\glt `(The kid) that (they) saw ran.'
\end{exe}

In Experiment 1, we tested the form hypothesis by comparing sentences
with verbal attractors to sentences with canonical nominal attractors in
Turkish. Experiment 2 then tested the form hypothesis more directly by
only using verbal attractors. We expected that if surface-overlap can
modulate relevant memory representations for dependency resolutions, we
would see similar attraction results with nominal and verbal attractors.
However, if participants are tracking an higher order cue that is
relevant for being a possible controller, then the verbal attractors,
due to their inability to control agreement, would not introduce
agreement attraction effects even though their high morpho-phonological
similarity.

Across both experiments, we found no evidence that verbal \emph{-lAr}
induces attraction, even when canonical nominal attractors are present
in the same session. This pattern aligns with prior findings in general
attraction literature and Turkish agreement attraction, namely
surface-form overlap alone does not derive agreement illusions. Rather,
attraction appears to depend on abstract feature overlap between
potential controllers and agreement probes, and possibly statistical
associations between the strings and their controllers. In this light,
findings of \citet{Slioussar2018} are best analyzed as a possible
increased association between genitive marking and possible subjecthood
and being an agreement controller, which is not possible in Czech, and
thus no attraction \citep{LacinaChromy2022}. By doing so, we hope to
clarify how cue-mechanisms are employed and the role of phonological
overlap in sentence processing.

\section{Experiment 1: Testing Surface-Form
Overlap}\label{experiment-1-testing-surface-form-overlap}

\subsection{Participants}\label{participants}

We recruited 95 undergraduate students to participate in the experiment
in exchange for course credit. Participants self-identified as native
Turkish speakers (0 non-native entries in metadata), with an average age
of 21 years (range: 18-30).

Preprocessing followed \texttt{exclude\_bad\_subjects\_8()}.
Subject-level screening used two discrimination checks:
\(\Delta_{gen} = p(yes\mid gen\_d) - p(yes\mid gen\_c)\) and
\(\Delta_{rc} = p(yes\mid rc\_d) - p(yes\mid rc\_c)\), with failure
defined as values less than or equal to 0.25. Under the current
conjunctive implementation, participants are excluded only if both
checks fail. In this sample, 1 participant(s) failed the gen check, 2
failed the rc check, and 0 failed both (excluded at subject level).

At the trial level, reaction-time trimming removed 229 trials (120 with
RT \textless= 200 ms; 109 with RT \textgreater= 4999 ms; 2.71\% of all
trials). Practice and missing-response trials were then removed (855
practice trials; 59 non-practice missing responses). The analyzed
dataset contained 95 participants and 7422 observations.

\subsection{Materials}\label{materials}

We used 40 sets of sentences like Table \ref{tab:stimuli_design}, in
which we manipulated (i) the number of the attractor, (ii) the type of
the attractor, and (iii) the number agreement on the verb. Both plural
markings were marked with the suffix \emph{-lAr}, while the singular
number and singular agreement were marked by its absence.

\begin{table}[ht]
\centering
\caption{Experimental conditions. The Attractor was manipulated for number and type. The Verb was manipulated to match or mismatch the head noun (always singular), creating Grammatical and Ungrammatical conditions.}
\label{tab:stimuli_design}

% Using \small to fit page width without scaling
\small 
\begin{tabular}{@{}ll l ll@{}}
\toprule
& & & \multicolumn{2}{c}{\textbf{Grammaticality (Verb Suffix)}} \\ \cmidrule(l){4-5} 
\textbf{Attr. Type} & \textbf{Attr. Num} & \textbf{Attractor} & \textbf{Grammatical} & \textbf{Ungrammatical (*)} \\ \midrule

% Verbal Section
% --- VERBAL SECTION ---
\multirow{4}{*}{\textbf{Verbal}} 
  & \multirow{2}{*}{SG} & Tut-tuğ-u & zıpla-dı & *zıpla-dı-lar \\
  & & \scriptsize \textit{hire-\textsc{nmlz-poss}} & \scriptsize \textit{jump-\textsc{pst}} & \scriptsize \textit{jump-\textsc{pst-pl}} \\ \cmidrule(l){2-5}
  
  & \multirow{2}{*}{PL} & Tut-tuk-lar-ı & zıpla-dı & *zıpla-dı-lar \\
  & & \scriptsize \textit{hire-\textsc{nmlz-pl-poss}} & \scriptsize \textit{jump-\textsc{pst}} & \scriptsize \textit{jump-\textsc{pst-pl}} \\ \midrule

% --- NOMINAL SECTION ---
\multirow{4}{*}{\textbf{Nominal}} 
  & \multirow{2}{*}{SG} & Milyoner-in & zıpla-dı & *zıpla-dı-lar \\
  & & \scriptsize \textit{millionaire-\textsc{gen}} & \scriptsize \textit{jump-\textsc{pst}} & \scriptsize \textit{jump-\textsc{pst-pl}} \\ \cmidrule(l){2-5}

  & \multirow{2}{*}{PL} & Milyoner-ler-in & zıpla-dı & *zıpla-dı-lar \\
  & & \scriptsize \textit{millionaire-\textsc{pl-gen}} & \scriptsize \textit{jump-\textsc{pst}} & \scriptsize \textit{jump-\textsc{pst-pl}} \\ \bottomrule
\end{tabular}

\vspace{1em} % Space between table and examples

\begin{exe}
    % \exsep controls space between examples if needed
    % \setlength{\exsep}{1ex} 
    
    \ex \textit{Verbal Attractor Conditions}
    % 1. Tighten space between Label (Line 1) and Sentence (Line 2)
    \vspace{-1.5ex} 
    \gll \textbf{{[}Attractor{]}} aşçı mutfak-ta sürekli \textbf{{[}Verb{]}}\\
         hire-\textsc{nmlz-(pl)-poss} cook kitchen-\textsc{loc} non.stop jump-\textsc{pst-(pl)}\\
    % 2. Tighten space between Gloss (Line 3) and Translation (Line 4)
    \vspace{-2ex} 
    \glt `The \textbf{[$_{Attr.}$~hired$_{pl}$/hired$_{sg}$]} cook \textbf{[$_{Verb}$~jumped$_{pl}$/jumped$_{sg}$]} in the kitchen non-stop.'\vspace{1ex} 

    \ex \textit{Nominal Attractor Conditions}
    \vspace{-1.5ex}
    \gll \textbf{{[}Attractor{]}} aşçı-sı mutfak-ta sürekli \textbf{{[}Verb{]}}\\
         millionaire-\textsc{(pl)-gen} cook-\textsc{poss} kitchen-\textsc{loc} non.stop jump-\textsc{pst-(pl)}\\
    \vspace{-2ex} 
    \glt `The \textbf{[$_{Attr.}$~millionaires'/millionaire's]} cook \textbf{[$_{Verb}$~jumped$_{pl}$/jumped$_{sg}$]} in the kitchen non-stop.'
\end{exe}

\end{table}

Verbal attractor conditions featured complex subject NPs containing a
bare head noun and a reduced relative clause acting as the attractor
(e.g., `tuttukları aşçı', `the hired cook'). Because nominal plural
marking is mandatory and the head noun was always singular, plural verb
agreement rendered these sentences ungrammatical. Nominal attractor
conditions, featuring nominal attractors such as `milyonerlerin aşçısı'
(`the millionaires' cook') were taken from \citet{TurkLogacev2024}. To
prevent participants from associating plural verbs with
ungrammaticality, fillers were balanced between grammatical sentences
with plural verbs and ungrammatical sentences with singular verbs.

\subsection{Procedures}\label{procedures}

The experiment was conducted online via Ibex Farm \citep{Drummond2013},
lasting approximately 25 minutes. After providing informed consent and
demographic details, participants read instructions and completed nine
practice trials.

Each trial began with a 600 ms blank screen, followed by a centered,
word-by-word RSVP presentation (30 pt font, 400 ms duration, 100 ms
inter-stimulus interval). Upon the prompt, participants judged sentence
acceptability as quickly as possible by pressing `P' (acceptable) or `Q'
(unacceptable). A red warning message appeared during practice
trials---but not experimental trials--if responses exceeded 5,000 ms.
Participants pressed the space bar to advance to the next item.

The study included 40 experimental and 40 filler sentences. Experimental
items were distributed across four lists using a Latin-square design,
ensuring each participant viewed only one list containing one version of
each item.

\subsection{Analysis and Results}\label{analysis-and-results}

Participants showed high accuracy in both grammatical (M = 0.95, CI =
{[}0.94,0.96{]}) and ungrammatical filler sentences (M = 0.06, CI =
{[}0.05,0.07{]}), indicating that they understood the task and performed
it reliably.

Figure~\ref{fig-exp1-condition-means} presents the overall means and
credible intervals for `yes' responses across experimental conditions,
as well as the previous data from \citet{TurkLogacev2024}, which is
quite similar to the magnitude of \citet{LagoEtAl2019}. As shown, in our
study, participant gave more `yes' responses to ungrammatical sentences
with plural genitive-marked nominal attractors (M = 0.12, CI =
{[}0.09,0.15{]}) compared to their singular counterparts (M = 0.12, CI =
{[}0.09,0.15{]}).

However, similar increase in acceptability was not found with relative
clause attractors (M = 0.05 and 0.05, CI = {[}0.03, 0.07{]} and {[}0.03,
0.07{]} for singular and plural attractors, respectively). Participants
rated grammatical sentences similarly independent of the attractor
number or attractor type.

\begin{figure}

\centering{

\pandocbounded{\includegraphics[keepaspectratio]{paper_files/figure-pdf/fig-exp1-condition-means-1.pdf}}

}

\caption{\label{fig-exp1-condition-means}Mean proportion of `acceptable'
responses by grammaticality, attractor number and attractor type. Error
bars show 95\% Clopper--Pearson confidence intervals.}

\end{figure}%

Our model-based analysis targeted the same question as the descriptive
results: whether verbal attractors induce attraction. We fitted a
Bayesian mixed-effects logistic regression to binary \emph{yes/no}
responses, combining the present dataset with the nominal-attractor
dataset from \citet{TurkLogacev2024}. The fixed-effects structure
included Grammaticality, Attractor Number, Attractor Type, and all
interactions; the random-effects structure included by-subject and
by-item intercepts and slopes justified by the design. Grammaticality
and Attractor Number were sum coded (grammatical = 0.5, ungrammatical =
-0.5; plural = 0.5, singular = -0.5). Attractor Type (Nominal-Current,
Nominal-TL24, Verbal) was encoded with two orthogonal Helmert contrasts:
\texttt{RC\_vs\_Gens} (Verbal vs.~the average of both nominal
conditions) and \texttt{GenCurrent\_vs\_GenTL24} (the two nominal
datasets against each other). This coding allows direct decomposition of
(i) attraction within each attractor type and (ii) between-type
differences in attraction magnitude.

We present posterior summaries of estimated regression effects from our
model in Figure~\ref{fig-exp1-fixed-effects}. Our model showed a robust
attraction in both nominal attractor cases, with strongly negative
effects for our nominal items (M = -1.45, CI = {[}-2.12, -0.78{]},
P(\textless0) = \textgreater0.99) and items from \citet{TurkLogacev2024}
(M = -1.16, CI = {[}-1.63, -0.69{]}, P(\textless0) = \textgreater0.99).
More importantly, our model found no evidence for an attraction in
verbal attractor conditions (M = 0.07, CI = {[}-0.73, 0.87{]},
P(\textless0) = 0.44), verifying our observations in the descriptive
statistics. We did not find an evidence for a difference in magnitude of
attraction between the two nominal-type attractors was not found (M =
-0.29, CI = {[}-1.11, 0.53{]}, P(\textless0) = 0.72), suggesting the
presence of an additional conditions did not affect attraction
magnitudes. Finally, we found strong evidence for a decreased overall
acceptability for nominal items in our experiment (M = -1.09, CI =
{[}-1.77, -0.44{]}, P(\textless0) = \textgreater0.99), suggesting the
within-experimental distribution did affect overall acceptability, but
not attraction.

\begin{figure}

\centering{

\pandocbounded{\includegraphics[keepaspectratio]{paper_files/figure-pdf/fig-exp1-fixed-effects-1.pdf}}

}

\caption{\label{fig-exp1-fixed-effects}Posterior summaries of
attraction-related effects. Points indicate posterior means, and
horizontal bars show 95\% credible intervals on the log-odds (β) scale.
Attraction was estimated as the interaction between grammaticality and
attractor number within each attractor type. Negative values indicate
stronger attraction (a reduced ungrammaticality penalty in
plural-attractor conditions). Dashed line denotes zero (no effect).}

\end{figure}%

\subsubsection{Bayes Factor Analysis for Null
Effects}\label{bayes-factor-analysis-for-null-effects}

To provide formal evidence for the absence of attraction with verbal
attractors, we computed Bayes Factors using the Savage-Dickey density
ratio method \citep{Wagenmakers2010}. This approach quantifies the
evidence for the null hypothesis (no effect) relative to the
alternative.

Bayes-factor computation for this section is temporarily deferred. We
will report BF\textsubscript{01} estimates for the verbal-vs-nominal
attraction contrast in a later revision.

\subsection{Discussion}\label{discussion}

Experiment 1 found no evidence that phonological overlap between nominal
and verbal plural morphemes in Turkish induces attraction. Participants
reliably rejected ungrammatical sentences with plural-marked verbal
attractors, contrasting with the canonical attraction effects observed
for nominal attractors. This indicates that the verbal plural marker
\emph{-lAr} does not generate interference comparable to nominal
plurals.

Our results and between-experiment comparisons indicate that
within-experiment statistics---specifically, exposure to verbal
attraction items---did not substantially reduce attraction magnitude.
However, overall acceptability for nominal attractor sentences was lower
than in \citet{TurkLogacev2024}. This aligns with prior work showing
that trial distributions modulate judgments. While previous studies
drove this effect via instructions or fillers
\citep{HammerlyEtAl2019, ArehalliWittenberg2021}, we demonstrate that
experimental conditions and the presence of an effect in a condition
subset also modulate overall acceptability, but surprisingly not the
attraction.

A potential concern is that our mixed design---combining canonical
nominal attractors with verbal ones---influenced response patterns. The
presence of robust nominal attraction may have altered participant
strategies, potentially masking weaker verbal effects
\citep{HammerlyEtAl2019, Turk2022}. To determine if the absence of
verbal attraction in Experiment 1 was genuine rather than a
distributional artifact, Experiment 2 removed all nominal attractors.
This design tests whether the null effect persists when verbal
morphology is the sole potential source of interference.

\subsubsection{Null-effect inference plan for Experiment
1}\label{null-effect-inference-plan-for-experiment-1}

Because the critical claim in Experiment 1 is a null effect for verbal
attractors, we will make the reporting workflow explicit after model
reruns. The goal is to show not only that a point-null is plausible, but
also that any remaining non-zero effect is too small to support an
attraction account.

We will add the following transparency details:

\begin{enumerate}
\def\labelenumi{\arabic{enumi}.}
\tightlist
\item
  \textbf{Procedure details.} We will report trial counts per condition,
  randomization/counterbalancing scheme, exclusion criteria with
  retained proportions, and whether any trial-level filtering changed
  the condition balance.
\item
  \textbf{Contrast definitions.} We will report the exact coding used in
  the model: Grammaticality and Attractor Number sum-coded at +/-0.5,
  and Attractor Type represented with two orthogonal Helmert contrasts
  (\texttt{RC\_vs\_Gens}, \texttt{GenCurrent\_vs\_GenTL24}).
\item
  \textbf{Model specification.} We will provide the fitted formula,
  priors, sampling settings, convergence checks (R-hat, ESS,
  divergences), and posterior predictive checks.
\item
  \textbf{Target estimand.} We will define verbal attraction as the
  model-implied Grammaticality x Attractor Number interaction within the
  verbal condition, computed from posterior draws (the \texttt{eff\_rc}
  quantity in the current analysis script).
\item
  \textbf{Null-effect evidence bundle.} We will report posterior mean
  and 95\% CrI for verbal attraction, BF\textsubscript{01} for the same
  estimand, posterior mass in a prespecified ROPE around zero, and a
  prior-sensitivity check (narrow, medium, wide priors).
\end{enumerate}

To keep this section concrete, we will use a short reporting template
once reruns are complete:

For verbal attractors, the attraction estimand was
\texttt{beta\ =\ {[}M{]}}, 95\% CrI \texttt{{[}L,\ U{]}}, with posterior
probability \texttt{P(beta\ \textless{}\ 0)\ =\ {[}p{]}}. A
Savage-Dickey test on the same estimand yielded
\texttt{BF\textasciitilde{}01\textasciitilde{}\ =\ {[}x{]}}, indicating
{[}strength{]} evidence for the null. Under prior-sensitivity analyses
({[}prior set 1{]}, {[}prior set 2{]}, {[}prior set 3{]}),
BF\textsubscript{01} remained in the {[}range{]} range. The posterior
mass inside the ROPE \texttt{{[}a,\ b{]}} was \texttt{{[}r{]}\%},
supporting the interpretation that any residual verbal-attractor effect
is practically negligible.

This fuller reporting makes the Experiment 1 null claim transparent and
sets up Experiment 2 as a planned test of robustness under a cleaner
design.

\section{Experiment 2: Isolating Verbal
Attractors}\label{experiment-2-isolating-verbal-attractors}

\subsection{Participants, Materials, and
Procedure}\label{participants-materials-and-procedure}

80 new undergraduate students who are native Turkish speakers (M = 21,
range: 18 -- 31) were recruited. We utilized the same verbal attractor
items and fillers from Experiment 1, removing all nominal attractor
trials. The experimental procedure was identical to Experiment 1.

\subsection{Analysis and Results}\label{analysis-and-results-1}

Participants showed high accuracy in both grammatical (M = 0.94, CI =
{[}0.92,0.95{]}) and ungrammatical filler sentences (M = 0.92, CI =
{[}0.9,0.93{]}), indicating that they understood the task and performed
it reliably.

Figure~\ref{fig-exp2-condition-means} presents the overall means and
credible intervals for `yes' responses across experimental conditions.
As shown, ungrammatical sentences with plural attractors were rated as
acceptable as their counterparts with singular attractors (M = 0.06 and
0.05, CI = {[}0.04, 0.07{]} and {[}0.03, 0.07{]} for singular and plural
attractors, respectively).

On the other hand, accuracy in grammatical conditions was modulated by
the number of the attractor in an unexpected way. Participants rated
grammatical sentences with singular attractors as grammatical less often
(M = 0.92, CI = {[}0.9,0.94{]}) compared to their counterpars with
plural attractors (M = 0.95, CI = {[}0.93,0.96{]}).

\begin{figure}

\centering{

\pandocbounded{\includegraphics[keepaspectratio]{paper_files/figure-pdf/fig-exp2-condition-means-1.pdf}}

}

\caption{\label{fig-exp2-condition-means}Mean proportion of `acceptable'
responses by grammaticality and attractor number. Error bars show 95\%
Clopper--Pearson confidence intervals.}

\end{figure}%

These descriptive trends were confirmed by our Bayesian mixed-effects
models implemented in brms, assuming a Bernoulli logit link. The model
was fitted to the binary \emph{yes/no} responses and included fixed
effects for Grammaticality and Attractor Number and their interaction,
and random intercepts and slopes for both subjects and items.

Posterior estimates are summarized in
Figure~\ref{fig-exp2-fixed-effects}. The model revealed a positive
effect of grammaticality (\(\beta\) = 5.92 {[}5.42, 6.46{]}, P(\(\beta\)
\textgreater{} 1.00)), but no reliable main effect of attractor number
(\(\beta\) = 0.15 {[}-0.19, 0.51{]}, P(\(\beta\) \textgreater{} 0.81)).
On the other hand, there was a small but positive interaction (\(\beta\)
= 0.67 {[}-0.01, 1.38{]}, P(\(\beta\) \textgreater{} 0.97)). To clarify
the effects' presence in grammaticals only, we fitted two more models
that is fitted to the subset of the data. While the model fitted to
grammatical conditions only showed an effect of attractor number
(\(\beta\) = 0.51 {[}0.06, 1.00{]}, P(\(\beta\) \textgreater{} 0.99)),
the model fitted to ungrammatical conditions, attraction relevant
conditions, did not provide evidence for the effect of number
manipulation (\(\beta\) = -0.05 {[}-0.45, 0.37{]}, P(\(\beta\)
\textgreater{} 0.99)). These results suggest that the presence of a
plural attractor did not increase the acceptability of ungrammatical
sentences, nor was this relationship modulated by grammaticality.

\begin{figure}

\centering{

\pandocbounded{\includegraphics[keepaspectratio]{paper_files/figure-pdf/fig-exp2-fixed-effects-1.pdf}}

}

\caption{\label{fig-exp2-fixed-effects}Posterior means and 95\% credible
intervals for fixed effects in the two Bayesian models. The x-axis shows
the posterior mean (log-odds scale). The blue intervals correspond to
the model in which a positive interaction was assumed, and the orange
intervals to the model in which it was not.}

\end{figure}%

\subsubsection{Bayes Factor Analysis for Null
Effects}\label{bayes-factor-analysis-for-null-effects-1}

To quantify evidence for the absence of attraction effects, we computed
Bayes Factors using the Savage-Dickey density ratio method
\citep{Wagenmakers2010}. This approach compares the posterior density at
the null value (zero) to the prior density at the same point, providing
a ratio of evidence for the null hypothesis (BF\textsubscript{01}).

Bayes-factor computation for this section is temporarily deferred. We
will report BF\textsubscript{01} estimates for the interaction and
main-effect tests in a later revision.

\subsection{Discussion}\label{discussion-1}

Experiment 2 replicated the verbal attractor conditions from Experiment
1 in isolation and again revealed no evidence for agreement attraction
driven by verbal plural markers. Ungrammatical sentences with plural
marked main verbs were rejected at similar rates regardless of whether
the reduced clause verb bore plural \emph{-lAr} or not, and there were
no reliable effects of attractor number or interactions involving
attractor number. This confirms that the absence of a verbal attraction
effect in Experiment 1 was not due to the presence of nominal attractor
items or to within experiment item statistics.

Unexpectedly, grammatical sentences with singular attractors were judged
less acceptable than those with plural attractors. This effect is
unlikely to reflect agreement attraction, since it arises in the
opposite direction. One possibility is that it results from an
interaction between plausibility and referential availability. The
plural morpheme can license a more general interpretation by allowing an
unspecific reference, whereas the singular reduced relative clause more
strongly invites a specific referent, which may be less accessible in
the context of the task. We do not pursue this explanation further, as
it falls outside the scope of the present paper.

\section{General Discussion}\label{general-discussion}

Summary of findings

Contextualizing of the findings

Theories of surface overlap

We investigated whether surface-overlap advantage seen in reading times
and comprehension questions can bleed into dependency resolution. Recent
work by \citet{Slioussar2018} argued that an accidental surface-overlap
with a nominative plural form may result in activation of relevant cues
even though the syntactic analysis of such a noun is clearly genitive
singular. However, modulation of agremeent-relevant cues seems to be
gated by being a possible controller in other relevant work in
syncretism, and similar manipulations in English and Czech were unable
to find a phonological modulation.

Using two speeded acceptability judgment experiments, we disentangled
the statistical property of being a controller from a surface overlap.
Turkish provides a useful test case because the plural \emph{-lAr}
appears both on verbs and on nouns, but only noun phrases can control
agreement. If phonological overlap alone can activate
controller-relevant cues, then plural-marked verbs in reduced relative
clauses should induce attraction effects even though they never control
agreement.

Across both experiments, we found that Turkish attraction is determined
by being a potential controller rather than merely resembling one.
Participants did not accepted ungrammatical sentences with containing
plural verbal attractors more often than their singular counterparts.
This absence of attraction persisted with or without a robust attraction
with nominal attractors in the same session.

These results indicate that attraction depends on abstract feature
overlap with potential controllers, not on surface-form similarity. This
pattern converges with prior results in English and Czech that failed to
find attraction for pseudoplural or phonologically plural forms
\citep{BockEberhard1993, HaskellMacDonald2003, NicolEtAl:2016, LacinaChromy2022},
but appears to stand in contrast to findings from Russian
\citep{Slioussar2018}.

While the most obvious difference is syntactic---our non-attracting
elements were verbs, whereas the attracting elements in Russian were
nouns \citep{Slioussar2018}---this distinction alone is insufficient, as
prior work shows that even pseudoplural nouns in English and the same
surface-overlap in Czech fail to attract
\citep{BockEberhard1993, LacinaChromy2022}. We propose instead that the
parser `gates' its search based on an element's abstract potential to be
a controller. The Russian genitive noun, despite its surface form, is
recognized as an element that can control agreement in other
constructions, thus passing this abstract gate. Our Turkish verbal
attractors or Czech genitive nouns, by contrast, lack this potential
entirely; they can never be controllers. They therefore fail this
gating, and no attraction is observed, despite the perfect phonological
overlap.

This interpretation aligns with cross-linguistic findings showing that
attraction is strongest when the attractor bears case or number
morphology that can be associated with subjects or agreement controllers
\citep{LagoEtAl2019, BhatiaDillon2022, BleotuDillon2024}. In other
words, it is not form overlap per se, but feature ambiguity or a
statistical association with controllerhood that matters. Earlier
formulations of these models left open whether `looking like' a
controller or `being able to be' a controller was critical. The present
high-powered results from Turkish favor the latter: only morphologically
licensed controllers, or those with a genuine abstract potential to be
one, engage in attraction.

\section{Appendix}\label{appendix}

\subsection*{Acknowledgment}\label{acknowledgment}
\addcontentsline{toc}{subsection}{Acknowledgment}

This project heavily benefited from discussions with Pavel Logacev. I am
also thankful first and foremost Ellen Lau, along with Colin Phillips,
Brian Dillon, and Radim Lacina for their comments on the manuscript.

\subsection*{Data availability}\label{data-availability}
\addcontentsline{toc}{subsection}{Data availability}

Materials, code and data available at: PSYARXIV LINK.

\section*{References}\label{references}
\addcontentsline{toc}{section}{References}

\renewcommand{\bibsection}{}
\bibliography{../bib/bibliography.bib}





\end{document}
